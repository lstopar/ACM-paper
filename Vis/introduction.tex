The visualization of multivariate time series is a challenging task. Often, systems are observed through one-dimensional measurements graphed over time. Modern systems often have many different sensors which 
typically operate in approximate cycles over time. Typical examples include weather systems (e.g. the seasons), manufacturing systems and power consumption. 
%
In this paper, we  present the StreamStory system for analyzing and visualizing multi-variate time series with minimal prior knowledge of the underlying time series. One of the key features of our system is that it allows for  \emph{multi-scale analysis} of the underlying system allowing the user to interactively find suitable scales to interpret their data in a qualitative representation.  

Intuitively, the system helps users detect behaviors which map to conceptual states of the system. For example, in the case of weather, a ``summer'' state may include higher average temperatures, while a ``winter" state would have lower temperatures. 
%
This representation is built using the following pipeline. First, we capture the structure of the data by employing unsupervised machine learning
techniques to identify the systems' most typical states. Next, we capture the
dynamics by modeling transitions between these states using a Markov chain framework.
Finally, we construct a hierarchical representation by aggregating both
states and transitions. Each level of the hierarchy is then associated with a unique
scale or detail level.

Overall, this system gives users a hierarchical view of the dynamics as well as 
allows for analysis of the system. After the initial model is built, we visualize the model
in a web-based user interface, (see Figure 1), allowing the 
user to zoom between scales interactively. To help users identify the
meaning of states and transitions, the system provides several automatic assistance services,
including automatic state labeling, rule extraction, attribute highlighting which
identify \emph{differences in states}, as well as several tools to help visualize an individual states'
structure. 
%
The system provides additional helper  services based on the Markov chain model. It can map online streaming data onto 
the model in real-time and issue alerts and messages based on parameters set in the interface. %This can be useful in several scenarios we describe in the paper.
%
The main technical contributions of the paper are
\begin{enumerate}{}
  \item A novel methodology for modeling and visualizing multivariate time-varying data streams. This approach is able to capture multiscale behaviour by building a hierarchical model of the system. We use Markov chains to visualize the dynamics of a system and propose an algorithm for constructing hierarchies of recurrent continuous time Markov chains, which preserves stationary distributions and has not appeared in the literature. %We also design our method to preserve as much of the dynamics as possible throughout the hierarchy.
  \item Building on these ideas, we develop and implement an interactive system which incorporates many different features to help analyze and understand complex systems through our visualization tool (\url{http://streamstory.ijs.si}).
  \item The system provides functionality to process and visualize online streaming data and provide the user with  alerts and messages on identified behaviours in the system.  
\end{enumerate}
\error{Should revise the types of contributions - Is it a visualization tool? Machine learning algorithm? If it's a visualization tool, what does it visualize?}

\error{3.3) Lack of discussion of the multivariate part. How does the system support multivariate visualization and how many variables does it support?}

\subsubsection{Possible Journals}

\begin{itemize}
	\item \lstopar{AI MAGAZINE}
	\item \lstopar{DATA MINING AND KNOWLEDGE DISCOVERY}
	\item \lstopar{ACM TRANSACTIONS ON GRAPHICS}
\end{itemize}