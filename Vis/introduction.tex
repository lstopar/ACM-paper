Sensory systems typically operate in cycles with a continuously time-varying component. These 
systems include, for example, the solar system, manufacturing systems or weather 
systems. Such systems can be characterized by a set of states, along with associated
state transitions. States, on a high level, may include a "day" state and a "night"
state or maybe states with high and low productivity. For example, when a pilot wishes
to change an aircrafts heading, they will put the aircraft into state "banking turn"
by lowering one aileron and raising the other, causing the aircraft to perform a
circular arc. After some time, the wings of the aircraft will be brought level by
an opposing motion of the ailerons and the aircraft will go back into state "level".

Such high-level states can be decomposed into lower-level states, giving us a 
multi-level view of the system and allowing us to observe the system on multiple 
detail levels. For example, a "banking turn" state can be decomposed by the
aircrafts roll and angular velocity, resulting in perhaps three states: "initiate
turn", "full turn" and "end turn".

We present a methodology for modeling such systems and demonstrate its implementation
called StreamStory. StreamStory is a tool for summarizing multivariate continuously
time-varying data streams as a hierarchy of Markovian processes by automatically learning
the typical low-level states, transitions and aggregating them into a hierarchy, obtaining
a unique Markovian process on multiple detail levels. As such it provides users with a 
summary of the structure and dynamics of the monitored system.

Furthermore, we divide the inputs streams into two sets: observation and control set.
Attributes in the observation set are the attributes that tell us the state of the 
system and, we assume, cannot directly influence its dynamics. These are parameters
that users cannot directly manipulate, like aircraft tilt from the previous example,
which must be indirectly manipulated through the angles of ailerons. We use observation
attributes to identify, and aggregate, low-level states, detect outliers (anomalies) 
and determine the current state of the system.

In contrast, users can directly manipulate attributes in the control set. These 
are attributes like angles of ailerons that may directly influence the behavior 
(observation attributes) and performance of the system. For example, when an operator 
in a steel factory sets the cooling temperature to a high value, the product will
take longer to go from state "hot" to state "cool". As such, we assume, control attributes
may influence the occurrence, and expected time, of undesired states, associated with
undesired events.

Our system uses control attributes to model state transitions, allowing us to observe
the dynamics with respect to the current configuration and gives us insight into the
expected dynamics with respect to some alternate attribute configuration.

The main research contributions presented in this paper can be summarized as follows:
\begin{enumerate}{}
  \item We present a novel methodology for modeling multivariate continuously time-varying data streams
  in a hierarchical manner providing a unique model of the several detail levels.
  \item We present a novel approach for modeling state transitions in a Markov chain allowing
  to observe the dynamics of the chain under alternate configurations.
  \item We propose a recursive algorithm for partitioning recurrent continuous time Markov chains.
\end{enumerate}


The remainder of this paper is structured as follows. In Section \ref{sec:methodology} we present
our methodology in detail. Section \ref{sec:ui} we present user interaction. Section \ref{sec:experiment}
we present our experimental results and finally in Section \ref{sec:conclusion} we conclude this
paper and give some ideas for further work.

