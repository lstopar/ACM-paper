\begin{itemize}
	\item Initial state construction (State Identification)
	\item Aggregation (State Aggregation)
	\item Transition probabililties (Modeling transitions)
\end{itemize}

We begin this section with an overview of our multi-level visualization methodology. The methodology
consists of three main steps: initial state construction, state aggregation and modeling transition
probabilities.

\subsection{Initial State Construction}

The first step in our methodology is the construction of initial lowest level states. Using the notation
from Section \ref{sec:preliminaries} we define a partition function 
from $\lstopar{f}: \mathbb{R}^d \rightarrow I$ mapping the $d$-dimensional signal to lowest level states
\lstopar{
with the following properties:
\begin{itemize}
	\item similar points in $\mathbb{R}^d$ should be mapped to the same or neighboring partitions.
\end{itemize} 
}

\subsection{Modeling Transition Probabilities}

\subsection{Aggregation}

Once we have computed the lowest level Markov chain $(X_t)_{t \ge 0}$, we need to be able to represent it on multiple scales.
As stated in Section \ref{sec:preliminaries} we associate each scale $s$ with a specific partition function
$\lstopar{f_s}: \mathbb{R}^d \rightarrow \lstopar{I_s}$ where coarser partitions are generated by merging 
states of finer partitions.
Suppose we have already computed the finest level partition $\lstopar{\cP}$, inducing state space $I$ and the
coarser partition at scale $\lstopar{\cP_s}$, inducing state space $J$. Suppose the finest level Markov chain,
on state space $I$ is represented by a transition rate matrix $Q$ and define a surjective partition function
$\phi: I \rightarrow J$. To compute the Markov chain induced by state space $J$, we adapt a formula proposed
in \cite{5746509} to continuous time Markov chains. We define a \lstopar{transition rate matrix transformation function}
$\Phi: \mathbb{R}^{n \times n} \rightarrow \mathbb{R}^{m \times m}$ using formula \ref{eq:ctmc-state-aggregation}.
\begin{equation}
	\label{eq:ctmc-state-aggregation}
	\Phi(Q) = \tilde{Q} = (P' \Pi P)^{-1} P' \Pi Q P
\end{equation}
where $\Pi = diag(\pi)$ and $\pi$ is the stationary distribution of $(X_t)_{t \ge 0}$, while $P$ is a 
$|I| \times |J|$ partition matrix with elements
\begin{equation}
	\nonumber
	\left(P\right)_{ij} = 
		\left\{
			\begin{array}{ll}
				1 & \mbox{if } \phi(i) = j \\
				0 & \mbox{otherwise}.
			\end{array}
		\right.
\end{equation}
If we define $\psi = \phi \circ \phi$, then Equation \ref{eq:ctmc-state-aggregation} can be rewritten as
\begin{equation}
	\nonumber
	\tilde{q}_{\phi(i)\phi(j)} = \frac{\sum\limits_{i \in \psi(i)}\pi_i \sum\limits_{j \in \psi(j)} q_{ij}}{\sum\limits_{i \in \psi(i)}\pi_i}
\end{equation}