\begin{itemize}
	\item \cite{5746509} present a recursive bi-partitioning algorithm for partitioning discrete-time Markov chains. The 
	algorithm uses Kullback-Leibler divergence rate to measure the difference between the original Markov
	model and the its approximation. The authors also present a formula for aggregating states in the discrete-time
	Markov chain.
	\item \cite{pande-beauchamp-bowman:2010:methods:markov-model-review} present an overview of using Markov chain models in the bio-chemical domain. The paper
	presents approaches for modeling states (assigning observations to states) as well as building a transition
	matrix (using counts). The authors mention that to improve transition estimation one can use Bayesian priors.
	For example a prior that includes the effect of detailed balance can enhance the effectiveness when dealing with
	small counts. The authors also mention simplifying the transition matrix into fewer states by looking at larger
	timescales. This is done by spectral clustering.
	\item \cite{Benz2004239} present a multi-resolution framework for analysis of image data by exploring a hierarchical
	image object network and representing strongly linked objects, using polygons and fuzzy systems.
\end{itemize}