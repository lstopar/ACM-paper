Large multivariate datasets have become common in many applications, including production lines,
monitoring systems, bioinformatics and social sciences. As the number of observations increases,
existing tools become cluttered and unresponsive. This clutter saturates visualization and hinders
data analysts as large response times make interactive exploration difficult.

Many techniques have been proposed in the literature that address this scalability problem.
They can generally be classified into two groups: data abstraction and clutter reduction 
techniques.

Techniques for data abstraction used in visualization include filtering [TODO cite], clustering
and sampling [TODO cite].

Clutter reduction assigns more space to interesting data and tries to hide less interesting data.
The most common techniques for clutter reduction include zooming and distortion.

This section starts by providing the state-of-the-art in clustering and Markov chain research and
then continues with an overview of existing visualization techniques.

\subsection{General Visualization Techniques}

Shneiderman \cite{545307} proposes a task-by-data-type taxonomy of information visualizations of multi-dimensional
data types as well as structured data which provides researchers and developers with a guideline for
the design and implementation of visualization systems. His work highlights the importance of data
abstraction operations such as summarization, filtering, zooming and extraction.

\subsection{Visualization in Data Mining}

Visualization is an important tool in data mining as it \textcolor{red}{...}. Data mining is commonly defined
as the extraction of patterns or models from data sets, usually as part of extracting high-level
knowledge from low-level data. Visualization is an important tool in this area as it gives analysts a
tool for exploring, understanding data and creating hypothesis.

Visualization techniques used in data mining range from simple scatter plots, box plots, heatmaps, etc.
To more complex structures such as PCA [TODO cite], decision trees [TODO cite] and dendrograms.

Oliveira and Levkowitz \cite{1207445} provide a survey of visualization tools used for data mining. Their survey focuses
to visualization of tabular data.

\subsection{Markov Chains}

Hierarchical structures are used in many real-world applications, as they are flexible in storing information
and allow the user to zoom into detail high detail levels, while hiding these details on the upper levels.

\begin{itemize}
	\item \cite{5746509} present a recursive bi-partitioning algorithm for partitioning discrete-time Markov chains. The 
	algorithm uses Kullback-Leibler divergence rate to measure the difference between the original Markov
	model and the its approximation. The authors also present a formula for aggregating states in the discrete-time
	Markov chain.
	
	\item \cite{pande-beauchamp-bowman:2010:methods:markov-model-review} present an overview of using Markov chain models in the bio-chemical domain. The paper
	presents approaches for modeling states (assigning observations to states) as well as building a transition
	matrix (using counts). The authors mention that to improve transition estimation one can use Bayesian priors.
	For example a prior that includes the effect of detailed balance can enhance the effectiveness when dealing with
	small counts. The authors also mention simplifying the transition matrix into fewer states by looking at larger
	timescales. This is done by spectral clustering.
	
	\item \cite{Benz2004239} present a multi-resolution framework for analysis of image data by exploring a hierarchical
	image object network and representing strongly linked objects, using polygons and fuzzy systems.
	
	\item \cite{4015421} Investigate data abstraction quality measures in multiresolution visualization systems. Spcifically, they propose
	a histogram difference measure and a Nearest neighbor measure to measure the quality of the abstraction.
	
	\item \textcolor{red}{[TODO Bertini and Santucci [7] present a quality measure for sampling and apply it to finding the optimal sampling level]}
	
	\item \cite{Fua:2000:SBM:614278.614457} Brushing consists of painting sections of the display using a 
	mouse or stylus to indicate the data elements to be selected. The paper presents a technique called
	structure-based brushing, which can be used to perform selection in data sets organized via hierarchical clustering
	and partitioning algorithms. Their brushing mechanism serves to select subsets of the hierarchical structure
	for further analysis. For the level of detail, they use either: cluster size, level number or the clusters' volume.
	They can then traverse the hierarchy based on this level.
	
	\item \cite{729555} present a technique called
	reconfigurable disc trees for visualizing large hierarchical data sets. Each node is associated with a disc around which 
	its children are placed. The node of the tree is located at the apex of a cone and all hte children are arranged around
	the circular base of hte cone.
	
	\item \textcolor{red}{Hierarchical Parallel Coordinates}
	
	\item \cite{Johnson:1991:TSA:949607.949654} present a visualization method for hierarchical structures called TreeMaps.
	TreeMaps makes 100\% use of the available space by mapping the full hierarchy onto a rectangular region in a space-filling
	manner, allowing large hierarchies to be displayed entirely, facilitating the presentation of semantic information.

	\item \cite{Elmqvist:2010:HAI:1749404.1749525}
	present a model for building, visualizing and interacting with multi-scale representations of visual information
	using hierarchical aggregation. Their work presents a model for transforming visualization techniques into a multiscale
	structure using hierarchical aggregation.
	
	Like their model, we also aggregate information in the data space.
	
	\item \textcolor{red}{Venn diagrams}
	
	\item \textcolor{red}{Categorical (hierarchical view)} [v brushes reference [12], [14]]
\end{itemize}

\textcolor{red}{We differ from the related work in that we summarize the information, instead of drawing the whole
dataset. We also summarize the dynamics of the dataset.}