
Visualization of large multi-variate time-series is an understudied area. In areas such as production and manufacturing lines, the common practice is to plot multiple signals in either the same window or in parallel windows~\cite{}. This makes interpretation difficult and does not scale particularly well to signals with a large number of variates or high sampling rates. %In the latter case, the appropriate time scale must be chosen a priori to enable a user to monitor the   
This clutters the visualization and hinders data analysis. Further, as the number of observations increases,
 large response times make interactive exploration difficult (citation?).

There is a large body of work to address this scalability problem.
They can generally be classified into two groups: data abstraction and clutter reduction 
techniques. Techniques for data abstraction include filtering [TODO cite], clustering
and sampling [TODO cite]. Clutter reduction assigns more space to interesting data and tries to hide less interesting data.
The most common techniques for clutter reduction include zooming and distortion [TODO cite].


We first overview some previoous work on multidimensional and hierarchical data visualizations. 
Shneiderman \cite{545307} proposes a task-by-data-type taxonomy of information visualizations of multi-dimensional
data types as well as structured data which provides researchers and developers with a guideline for
the design and implementation of visualization systems. His work highlights the importance of data
abstraction operations such as summarization, filtering, zooming and extraction.

Jeong and Pang \cite{729555} present a technique called reconfigurable disc trees for visualizing large 
hierarchical data sets. The data is presented as a tree which can be laid out in two or three dimensions.
Their visualization is based on discs around which the children of each node are placed and eliminates
visual overlaps among subtrees.

Johnson and Shneiderman \cite{Johnson:1991:TSA:949607.949654} present another technique for visualizing hierarchical
information called TreeMaps. TreeMaps recursively partition the display into rectangular bounding boxes representing the 
tree structure. The drawing of nodes within their bounding box is dependent on their content and can be interactively
controlled.

Fua et. al. \cite{Fua:2000:SBM:614278.614457} present a mechanism for navigating hierarchically organized structures
called Structure-Based brushes. Brushing consists of painting sections of the display using a mouse or stylus, indicating
data items to be selected. Their technique can be used to perform selection in datasets organized via hierarchical
clustering and partitioning algorithms and serves to perform subset selection for further analysis. The technique defines
the level of detail which can be e.g. the cluster size, volume of the level number which can be used to traverse the
hierarchy.

Elmqvist et. al. \cite{Elmqvist:2010:HAI:1749404.1749525} present a model for building, visualizing and interacting with
multi-scale representations of visual information using hierarchical aggregation. Their model allows for augmenting visualization
techniques into a multi-scale structure using hierarchical aggregation.

In their work, Cui et. al. \cite{4015421} investigate data abstraction quality measures in multi-resolution
visualization systems. Specifically, they propose a histogram difference measure and a Nearest neighbor
measure to measure the quality of the abstraction.

On important distinction between this work and ours is that this methods are not tailored to time-series data.


Visualization in data mining or/and exploratory data analysis range from simple techniques for data exploration, such as
scatter plots, box plots, heatmaps, etc., to visualization of more complex structures such as PCA \lstopar{[TODO cite]},
association rules, decision trees \lstopar{[TODO cite]}, clusters and dendrograms.

Oliveira and Levkowitz \cite{1207445} provide a survey of visualization tools used for data mining. Their survey focuses
to visualization of tabular data and provides an overview of tools and techniques for data exploration,
visualization of the extracted knowledge, discussing the question of how to select an appropriate visualization
technique.

Keim \cite{981847} provides a classification of information visualization techniques used in data mining, which is based on
the data type, visualization technique and interaction and distortion technique.

Benz et. al. \cite{Benz2004239} present a multi-resolution framework for analysis of image data, by exploring a hierarchical
image object network and representing strongly linked objects using polygons and fuzzy systems.


We base our techniques on  Markov chains. Markov chains are a special kind of memoryless stochastic process and typically only consist of a countable number of states. These two properties make them useful for many applications as they allow the computation of predictions as
well as quantifying their behavior by calculating probabilities and expected values. Markov chains are widely iused for modelling systems dating back to h Andrey Markov himself \cite{markov13} in 1913. Since then
Markov chains have had many famous applications such as Shannon's application to information theory \cite{Shannon:1948},
Baum's Hidden Markov Models (HMM) \cite{baum1970} and the application to web search by Sergey Brin and Larry Page \cite{Lawrence981}.
%
Other applications include the simulation of protein folding \cite{pande-beauchamp-bowman:2010:methods:markov-model-review},
analysis of biochemical networks \cite{Ciocchetta2009145},
genetics \cite{Huelsenbeck2310}, sensor networks \cite{DBLP:journals/corr/AlsheikhHNTL15}, agriculture, queuing theory, etc.

There has been some work on learning Markov chains. In their paper, Deng et. al. \cite{5746509} explored the aggregation of Markov chains using an information-theoretic
approach. Their work focuses on model reduction of complex discrete time Markov chains, using Kullback-Leibler divergence
rate to measure the difference between the original model and the approximation and provides a recursive bi-partitioning
algorithm for partitioning discrete time Markov chains. Their work provides a state aggregation solution by averaging
transition probabilities based on the stationary distribution. In our work we adapt this solution to the continuous time
setting.

There has also been a large body on work on approximate Markov chains for Monte Carlo Markov Chain (MCMC) simulation [todo cite]. In this case, however, the emphasis is on reducing the mixing time to ensure fast convergence. In our case, we want to preserve as much of the dynamics as possible. 


\begin{comment}
Large multivariate datasets have become common in many applications, including production lines,
monitoring systems, bioinformatics and social sciences. As the number of observations increases,
existing tools become cluttered and unresponsive. This clutter saturates visualization and hinders
data analysts as large response times make interactive exploration difficult.

Many techniques have been proposed in the literature that address this scalability problem.
They can generally be classified into two groups: data abstraction and clutter reduction 
techniques.

Techniques for data abstraction include filtering [TODO cite], clustering
and sampling [TODO cite].

Clutter reduction assigns more space to interesting data and tries to hide less interesting data.
The most common techniques for clutter reduction include zooming and distortion.

This section starts by providing the state-of-the-art in clustering and Markov chain research and
then continues with an overview of existing visualization techniques.



\subsection{General Visualization Techniques}


Shneiderman \cite{545307} proposes a task-by-data-type taxonomy of information visualizations of multi-dimensional
data types as well as structured data which provides researchers and developers with a guideline for
the design and implementation of visualization systems. His work highlights the importance of data
abstraction operations such as summarization, filtering, zooming and extraction.

Jeong and Pang \cite{729555} present a technique called reconfigurable disc trees for visualizing large 
hierarchical data sets. The data is presented as a tree which can be laid out in two or three dimensions.
Their visualization is based on discs around which the children of each node are placed and eliminates
visual overlaps among subtrees.

Johnson and Shneiderman \cite{Johnson:1991:TSA:949607.949654} present another technique for visualizing hierarchical
information called TreeMaps. TreeMaps recursively partition the display into rectangular bounding boxes representing the 
tree structure. The drawing of nodes within their bounding box is dependent on their content and can be interactively
controlled.

Fua et. al. \cite{Fua:2000:SBM:614278.614457} present a mechanism for navigating hierarchically organized structures
called Structure-Based brushes. Brushing consists of painting sections of the display using a mouse or stylus, indicating
data items to be selected. Their technique can be used to perform selection in datasets organized via hierarchical
clustering and partitioning algorithms and serves to perform subset selection for further analysis. The technique defines
the level of detail which can be e.g. the cluster size, volume of the level number which can be used to traverse the
hierarchy.

Elmqvist et. al. \cite{Elmqvist:2010:HAI:1749404.1749525} present a model for building, visualizing and interacting with
multi-scale representations of visual information using hierarchical aggregation. Their model allows for augmenting visualization
techniques into a multi-scale structure using hierarchical aggregation.

In their work, Cui et. al. \cite{4015421} investigate data abstraction quality measures in multi-resolution
visualization systems. Specifically, they propose a histogram difference measure and a Nearest neighbor
measure to measure the quality of the abstraction.

\subsection{Visualization in Data Mining}

Visualization is an important tool in data mining as it \textcolor{red}{...}. Data mining is commonly defined
as the extraction of patterns or models from data sets, usually as part of extracting high-level
knowledge from low-level data. Visualization is an important tool in this area as it gives analysts a
tool for exploring, understanding data and creating hypothesis.

Visualization techniques used in data mining range from simple techniques for data exploration, such as
scatter plots, box plots, heatmaps, etc., to visualization of more complex structures such as PCA \lstopar{[TODO cite]},
association rules, decision trees \lstopar{[TODO cite]}, clusters and dendrograms.

Oliveira and Levkowitz \cite{1207445} provide a survey of visualization tools used for data mining. Their survey focuses
to visualization of tabular data and provides an overview of tools and techniques for data exploration,
visualization of the extracted knowledge, discussing the question of how to select an appropriate visualization
technique.

Keim \cite{981847} provides a classification of information visualization techniques used in data mining, which is based on
the data type, visualization technique and interaction and distortion technique.

Benz et. al. \cite{Benz2004239} present a multi-resolution framework for analysis of image data, by exploring a hierarchical
image object network and representing strongly linked objects using polygons and fuzzy systems.

\subsection{Markov Chains}

Markov chains are a special kind of stochastic process
which can assume only a finite or countable number of states and have no memory of where they have been in the past.
These two properties make them useful for many applications as they allow the computation of predictions as
well as quantifying their behavior by calculating probabilities and expected values.

The first application of Markov chains begins with Andrey Markov himself \cite{markov13} in 1913. Since then
Markov chains have had many famous applications such as Shannon's application to information theory \cite{Shannon:1948},
Baum's Hidden Markov Models (HMM) \cite{baum1970} and the application to web search by Sergey Brin and Larry Page \cite{Lawrence981}.

Other applications include the simulation of protein folding \cite{pande-beauchamp-bowman:2010:methods:markov-model-review},
analysis of biochemical networks \cite{Ciocchetta2009145},
genetics \cite{Huelsenbeck2310}, sensor networks \cite{DBLP:journals/corr/AlsheikhHNTL15}, agriculture, queuing theory, etc.

In their paper, Deng et. al. \cite{5746509} explored the aggregation of Markov chains using an information-theoretic
approach. Their work focuses on model reduction of complex discrete time Markov chains, using Kullback-Leibler divergence
rate to measure the difference between the original model and the approximation and provides a recursive bi-partitioning
algorithm for partitioning discrete time Markov chains. Their work provides a state aggregation solution by averaging
transition probabilities based on the stationary distribution. In our work we adapt this solution to the continuous time
setting.

\lstopar{Hierarchical structures are used in many real-world applications, as they are flexible in storing information
and allow the user to zoom into detail high detail levels, while hiding these details on the upper levels.}

\lstopar{We differ from the related work in that we summarize the information, instead of drawing the whole
dataset. We also summarize the dynamics of the dataset.}
\end{comment}