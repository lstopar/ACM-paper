Large multivariate datasets have become common in many applications, including production lines,
monitoring systems, bioinformatics and social sciences. As the number of observations increases,
existing tools become cluttered and unresponsive. This clutter saturates visualization and hinders
data analysts as large response times make interactive exploration difficult.

Many techniques have been proposed in the literature that address this scalability problem.
They can generally be classified into two groups: data abstraction and clutter reduction 
techniques.

Techniques for data abstraction used in visualization include filtering [TODO cite], clustering
and sampling [TODO cite].

Clutter reduction assigns more space to interesting data and tries to hide less interesting data.
The most common techniques for clutter reduction include zooming and distortion.

This section starts by providing the state-of-the-art in clustering and Markov chain research and
then continues with an overview of existing visualization techniques.

\subsection{General Visualization Techniques}

Shneiderman \cite{545307} proposes a task-by-data-type taxonomy of information visualizations of multi-dimensional
data types as well as structured data which provides researchers and developers with a guideline for
the design and implementation of visualization systems. His work highlights the importance of data
abstraction operations such as summarization, filtering, zooming and extraction.

Jeong and Pang \cite{729555} present a technique called reconfigurable disc trees for visualizing large 
hierarchical data sets. The data is presented as a tree which can be laid out in two or three dimensions.
Their visualization is based on discs around which the children of each node are placed and eliminates
visual overlaps among subtrees.

Johnson and Shneiderman \cite{Johnson:1991:TSA:949607.949654} present another technique for visualizing hierarchical
information called TreeMaps. TreeMaps recursively partition the display into rectangular bounding boxes representing the 
tree structure. The drawing of nodes within their bounding box is dependent on their content and can be interactively
controlled.

Fua et. al. \cite{Fua:2000:SBM:614278.614457} present a mechanism for navigating hierarchically organized structures
called Structure-Based brushes. Brushing consists of painting sections of the display using a mouse or stylus, indicating
data items to be selected. Their technique can be used to perform selection in datasets organized via hierarchical
clustering and partitioning algorithms and serves to perform subset selection for further analysis. The technique defines
the level of detail which can be e.g. the cluster size, volume of the level number which can be used to traverse the
hierarchy.

Elmqvist et. al. \cite{Elmqvist:2010:HAI:1749404.1749525} present a model for building, visualizing and interacting with
multi-scale representations of visual information using hierarchical aggregation. Their model allows for augmenting visualization
techniques into a multi-scale structure using hierarchical aggregation.

\subsection{Visualization in Data Mining}

Visualization is an important tool in data mining as it \textcolor{red}{...}. Data mining is commonly defined
as the extraction of patterns or models from data sets, usually as part of extracting high-level
knowledge from low-level data. Visualization is an important tool in this area as it gives analysts a
tool for exploring, understanding data and creating hypothesis.

Visualization techniques used in data mining range from simple techniques for data exploration, such as
scatter plots, box plots, heatmaps, etc., to visualization of more complex structures such as PCA [TODO cite],
association rules, decision trees [TODO cite], clusters and dendrograms.

Oliveira and Levkowitz \cite{1207445} provide a survey of visualization tools used for data mining. Their survey focuses
to visualization of tabular data and provides an overview of tools and techniques for data exploration,
visualization of the extracted knowledge, discussing the question of how to select an appropriate visualization
technique.

Keim \cite{981847} provides a classification of information visualization techniques used in data mining, which is based on
the data type, visualization technique and interaction and distortion technique.

\subsection{Markov Chains}

Hierarchical structures are used in many real-world applications, as they are flexible in storing information
and allow the user to zoom into detail high detail levels, while hiding these details on the upper levels.

\begin{itemize}
	\item \cite{5746509} present a recursive bi-partitioning algorithm for partitioning discrete-time Markov chains. The 
	algorithm uses Kullback-Leibler divergence rate to measure the difference between the original Markov
	model and the its approximation. The authors also present a formula for aggregating states in the discrete-time
	Markov chain.
	
	\item \cite{pande-beauchamp-bowman:2010:methods:markov-model-review} present an overview of using Markov chain models in the bio-chemical domain. The paper
	presents approaches for modeling states (assigning observations to states) as well as building a transition
	matrix (using counts). The authors mention that to improve transition estimation one can use Bayesian priors.
	For example a prior that includes the effect of detailed balance can enhance the effectiveness when dealing with
	small counts. The authors also mention simplifying the transition matrix into fewer states by looking at larger
	timescales. This is done by spectral clustering.
	
	\item \cite{Benz2004239} present a multi-resolution framework for analysis of image data by exploring a hierarchical
	image object network and representing strongly linked objects, using polygons and fuzzy systems.
	
	\item \cite{4015421} Investigate data abstraction quality measures in multiresolution visualization systems. Spcifically, they propose
	a histogram difference measure and a Nearest neighbor measure to measure the quality of the abstraction.
	
	\item \textcolor{red}{[TODO Bertini and Santucci [7] present a quality measure for sampling and apply it to finding the optimal sampling level]}
	
	\item \textcolor{red}{Hierarchical Parallel Coordinates}
	
	Like their model, we also aggregate information in the data space.
	
	\item \textcolor{red}{Venn diagrams}
	
	\item \textcolor{red}{Categorical (hierarchical view)} [v brushes reference [12], [14]]
\end{itemize}

\textcolor{red}{We differ from the related work in that we summarize the information, instead of drawing the whole
dataset. We also summarize the dynamics of the dataset.}