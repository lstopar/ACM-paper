\subsection{Prediction Evaluation}

This experiment is intended to test the validity of our model. It was conducted on simulated data of 
an electric motor. When a button is initially pressed, the motor is turned on and ...

\begin{figure}[h!]
	\centering
	\begin{subfigure}{.5\columnwidth}
	  	\centering
	  	\includegraphics[width=\columnwidth]{simulation-processed}
  		\caption{\lstopar{TODO}}
  		\label{fig:simulation}
	\end{subfigure}
	\begin{subfigure}{.5\columnwidth}
	  	\centering
	  	\includegraphics[width=\columnwidth]{model-motor-simulated}
	  	\caption{\lstopar{TODO}}
	  	\label{fig:example-motor}
	\end{subfigure}
\end{figure}

\subsection{Weather Data}
\label{sec:experiments-weather}

The example below shows our model generated on monthly rainfall and temperature data
collected over the course of 20 years between 1920 and 1940 in Nottingham England.

\begin{figure}[h!]
	\centering
	\includegraphics[width=\columnwidth]{example-weather}
	\caption{Qualitative representation of temperature and rainfall data collected over the course of 20 years.}
	\label{fig:example-weather}
\end{figure}

The model was generated using the raw rainfall and temperature data, but each feature vector includes
the rainfall and temperature of the previous month. The states on the right hand side represent the 
summer states, while the states on the left represent winter states. The yearly timeline flows in the
counter clockwise direction with the spring states residing on the bottom of the figure and the autumn
states on the top.

Interestingly, in this dataset, the rainfall and temperature are very highly correlated and the auto naming feature
choose high rainfall as the most significant feature of the summer states. This correlation can be seen
from the attribute histogram shown in Figure \ref{fig:histograms-summer}.

\begin{figure}[h!]
	\centering
	\includegraphics[width=0.5\columnwidth]{histograms-summer}
	\caption{\textcolor{red}{TODO caption}}
	\label{fig:histograms-summer}
\end{figure}

\subsection{GPS Data}

The second example was created using raw GPS coordinates collected using a smartphone between \textcolor{red}{X} and \textcolor{red}{Y}.
The data represents the everyday movement of a European computer science researcher. Figure \ref{fig:example-geo}
shows our qualitative representation of this data on a high level with 8 state.

\begin{figure}[h!]
	\centering
	\includegraphics[width=\columnwidth]{geo-states}
	\caption{\textcolor{red}{TODO caption}}
	\label{fig:example-geo}
\end{figure}

From the figure, we can see that the system was able to identify the most typical locations of this persons
movements. Most of the time they spend in central EU including two small states on the bottom and right
representing southern Europe and India where they went for vacation. On the European continent, the system
also identified Germany and Sweden where the person frequently attends meetings or stays for a short while
during a connecting flight. The states on the left represent the USA with the largest state representing New
York city where the person spent the 2014 summer and the smaller states representing San Francisco and Austin,
Texas.

\subsection{Domain Experts}