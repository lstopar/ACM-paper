\documentclass[journal]{vgtc}                % final (journal style)
%\documentclass[review,journal]{vgtc}         % review (journal style)
%\documentclass[widereview]{vgtc}             % wide-spaced review
%\documentclass[preprint,journal]{vgtc}       % preprint (journal style)
%\documentclass[electronic,journal]{vgtc}     % electronic version, journal

%% Uncomment one of the lines above depending on where your paper is
%% in the conference process. ``review'' and ``widereview'' are for review
%% submission, ``preprint'' is for pre-publication, and the final version
%% doesn't use a specific qualifier. Further, ``electronic'' includes
%% hyperreferences for more convenient online viewing.

%% Please use one of the ``review'' options in combination with the
%% assigned online id (see below) ONLY if your paper uses a double blind
%% review process. Some conferences, like IEEE Vis and InfoVis, have NOT
%% in the past.

%% Please note that the use of figures other than the optional teaser is not permitted on the first page
%% of the journal version.  Figures should begin on the second page and be
%% in CMYK or Grey scale format, otherwise, colour shifting may occur
%% during the printing process.  Papers submitted with figures other than the optional teaser on the
%% first page will be refused.

%% These three lines bring in essential packages: ``mathptmx'' for Type 1
%% typefaces, ``graphicx'' for inclusion of EPS figures. and ``times''
%% for proper handling of the times font family.

\usepackage{mathptmx}
\usepackage{graphicx}
\usepackage{times,amsfonts,amssymb,amsthm}
\usepackage{times}
\usepackage{amsfonts}
\usepackage{amsthm}
\usepackage{amsmath}
\usepackage{subcaption}
\usepackage{comment}

\theoremstyle{definition}
\newtheorem{defn}{Definition} % definition numbers are dependent on theorem numbers
\newtheorem{theorem}{Theorem} % definition numbers are dependent on theorem numbers

%% commands
\newcommand{\cP}{\mathcal{P}}
\newcommand{\R}{\mathbb{R}}
\newcommand{\cM}{\mathcal{M}}
\newcommand{\primoz}[1]{\textcolor{red}{#1}}
\newcommand{\lstopar}[1]{\textcolor{blue}{#1}}
\newcommand{\dunja}[1]{\textcolor{green}{#1}}

\newcommand{\argmin}{\operatornamewithlimits{argmin}}
\DeclareMathOperator{\Ima}{Im}

\graphicspath{{img/}}

%% We encourage the use of mathptmx for consistent usage of times font
%% throughout the proceedings. However, if you encounter conflicts
%% with other math-related packages, you may want to disable it.

%% This turns references into clickable hyperlinks.
\usepackage[bookmarks,backref=true,linkcolor=black]{hyperref} %,colorlinks
\hypersetup{
  pdfauthor = {},
  pdftitle = {},
  pdfsubject = {},
  pdfkeywords = {},
  colorlinks=true,
  linkcolor= black,
  citecolor= black,
  pageanchor=true,
  urlcolor = black,
  plainpages = false,
  linktocpage
}

%% If you are submitting a paper to a conference for review with a double
%% blind reviewing process, please replace the value ``0'' below with your
%% OnlineID. Otherwise, you may safely leave it at ``0''.
\onlineid{0}

%% declare the category of your paper, only shown in review mode
\vgtccategory{Research}

%% allow for this line if you want the electronic option to work properly
\vgtcinsertpkg

%% In preprint mode you may define your own headline.
%\preprinttext{To appear in an IEEE VGTC sponsored conference.}

%% Paper title.

\title{StreamStory: Exploring Multivariate Time Series on Multiple Scales}

%% This is how authors are specified in the journal style

%% indicate IEEE Member or Student Member in form indicated below
\author{Luka Stopar and Primoz Skraba}
\authorfooter{
%% insert punctuation at end of each item
\item
Luka Stopar is with Jozef Stefan Institute. E-mail: luka.stopar@ijs.si.
\item
Primoz Skraba is with Jozef Stefan Institute. E-mail: primoz.skraba@ijs.si.
}

%other entries to be set up for journal
\shortauthortitle{Stopar \MakeLowercase{\textit{et al.}}:StreamStory: A Multi-Level Methodology for Explaining Data Streams}
%\shortauthortitle{Firstauthor \MakeLowercase{\textit{et al.}}: Paper Title}

%% Abstract section.
\abstract{
	This paper presents a novel multi-scale methodology for modeling, visualization and summarization of a collection
	of continuously time-varying data streams. The proposed methodology represents such a collection in qualitative 
	manner using a novel hierarchical framework by partitioning and aggregating the data streams using unsupervised data mining
	methods and constructing a Markovian transition model, capturing the dynamics of the data. The resulting model
	provides a visual summary and allows users to explore the data on several scales.  We validate our methodology on
	three real-world datasets.
} % end of abstract

%% Keywords that describe your work. Will show as 'Index Terms' in journal
%% please capitalize first letter and insert punctuation after last keyword
\keywords{Visualization, Time series, Multi-scale, Summarization, Markov chains, Exploratory data analysis}

%% ACM Computing Classification System (CCS). 
%% See <http://www.acm.org/class/1998/> for details.
%% The ``\CCScat'' command takes four arguments.

\CCScatlist{ % not used in journal version
 \CCScat{K.6.1}{Management of Computing and Information Systems}%
{Project and People Management}{Life Cycle};
 \CCScat{K.7.m}{The Computing Profession}{Miscellaneous}{Ethics}
}



% Uncomment below to include a teaser figure.
\teaser{
 	\centering
 	\includegraphics[width=16cm]{teaser}
	\caption{
  \label{fig:teaser}  
	On the move: yearly travels of a European researcher. A multi-scale summary of GPS coordinates collected over the course of four years using a smartphone. Our system summarizes the dataset in a qualitative manner using states and transitions and models them in a hierarchical manner, allowing users to traverse the hierarchy using the zoom function. From the visual summary we were able to identify the typical locations the person visited, including the USA, Germany, China, Italy and Slovenia.
	This is the main three panel user interface of the system.  The central panel shows the main visualization of the dynamics through a Markov chain model. The side and bottom panels provide additional information on the state which has been selected in the center panel. Here, the selected state is NY, corresponding to New York City. This can be inferred using the average geocoordinates in the right panel. In the bottom panel we see a timeline of when the system was in that state.  We see several distinct short trips to NY, with one longer stay shown in the center.
%	Our interface has several different views illustrating different aspects of the data.
	}
}

%% Uncomment below to disable the manuscript note
%\renewcommand{\manuscriptnotetxt}{}

%% Copyright space is enabled by default as required by guidelines.
%% It is disabled by the 'review' option or via the following command:
% \nocopyrightspace

%%%%%%%%%%%%%%%%%%%%%%%%%%%%%%%%%%%%%%%%%%%%%%%%%%%%%%%%%%%%%%%%
%%%%%%%%%%%%%%%%%%%%%% START OF THE PAPER %%%%%%%%%%%%%%%%%%%%%%
%%%%%%%%%%%%%%%%%%%%%%%%%%%%%%%%%%%%%%%%%%%%%%%%%%%%%%%%%%%%%%%%%

\begin{document}

%% The ``\maketitle'' command must be the first command after the
%% ``\begin{document}'' command. It prepares and prints the title block.
\firstsection{Introduction}

\maketitle

%% the only exception to this rule is the \firstsection command
%\section{Introduction}
\label{sec:introduction}
The visualization of multivariate time series is a challenging task. Often, systems are observed through one-dimensional measurements graphed over time. Modern systems often have many different sensors which 
typically operate in approximate cycles over time. Typical examples include weather systems (e.g. the seasons), manufacturing systems and power consumption. 
%
In this paper, we  present the StreamStory system for analyzing and visualizing multi-variate time series with minimal prior knowledge of the underlying time series. One of the key features of our system is that it allows for  \emph{multi-scale analysis} of the underlying system allowing the user to interactively find suitable scales to interpret their data in a qualitative representation.  

Intuitively, the system helps users detect behaviors which map to conceptual states of the system. For example, in the case of weather, a ``summer'' state may include higher average temperatures, while a ``winter" state would have lower temperatures. 
%
This representation is built using the following pipeline. First, we capture the structure of the data by employing unsupervised machine learning
techniques to identify the systems' most typical states. Next, we capture the
dynamics by modeling transitions between these states using a Markov chain framework.
Finally, we construct a hierarchical representation by aggregating both
states and transitions. Each level of the hierarchy is then associated with a unique
scale or detail level.

Overall, this system gives users a hierarchical view of the dynamics as well as 
allows for analysis of the system. After the initial model is built, we visualize the model
in a web-based user interface, (see Figure 1), allowing the 
user to zoom between scales interactively. To help users identify the
meaning of states and transitions, the system provides several automatic assistance services,
including automatic state labeling, rule extraction, attribute highlighting which
identify \emph{differences in states}, as well as several tools to help visualize an individual states'
structure. 
%
The system provides additional helper  services based on the Markov chain model. It can map online streaming data onto 
the model in real-time and issue alerts and messages based on parameters set in the interface. %This can be useful in several scenarios we describe in the paper.
%
The main technical contributions of the paper are
\begin{enumerate}{}
  \item A novel methodology for modeling and visualizing multivariate time-varying data streams. This approach is able to capture multiscale behaviour by building a hierarchical model of the system. We use Markov chains to visualize the dynamics of a system and propose an algorithm for constructing hierarchies of recurrent continuous time Markov chains, which preserves stationary distributions and has not appeared in the literature. %We also design our method to preserve as much of the dynamics as possible throughout the hierarchy.
  \item Building on these ideas, we develop and implement an interactive system which incorporates many different features to help analyze and understand complex systems through our visualization tool (\url{http://streamstory.ijs.si}).
  \item The system provides functionality to process and visualize online streaming data and provide the user with  alerts and messages on identified behaviours in the system.  
\end{enumerate}
\error{Should revise the types of contributions - Is it a visualization tool? Machine learning algorithm? If it's a visualization tool, what does it visualize?}

\error{3.3) Lack of discussion of the multivariate part. How does the system support multivariate visualization and how many variables does it support?}

\subsubsection{Possible Journals}

\begin{itemize}
	\item \lstopar{AI MAGAZINE}
	\item \lstopar{DATA MINING AND KNOWLEDGE DISCOVERY}
	\item \lstopar{ACM TRANSACTIONS ON GRAPHICS}
\end{itemize}


%% \section{Introduction} %for journal use above \firstsection{..} instead

% \begin{itemize}
% 	\item podaljsaj captione - naj bodo samozadostni (8 skoraj vredu)
% 	\item slika - zrihtaj state in transi., daj a, b ,c d pod puscice
% 	\item figure 5 malo vecju, daj enega nad drugega
% 	\item \textcolor{red}{Nekje je treba omeniti MDS}
% \end{itemize}

\section{Previous Work}
\label{sec:previous}

Visualization of large multi-variate time-series is an understudied area. In areas such as production and manufacturing lines, the common practice is to plot multiple signals in either the same window or in parallel windows~\cite{} \lstopar{[TODO ref missing]}. This makes interpretation difficult and does not scale particularly well to signals with a large number of variates or high sampling rates. %In the latter case, the appropriate time scale must be chosen a priori to enable a user to monitor the   
This clutters the visualization and hinders data analysis. Further, as the number of observations increases,
 large response times make interactive exploration difficult \lstopar{(citation?)}.

There is a large body of work to address this scalability problem.
They can generally be classified into two groups: data abstraction and clutter reduction 
techniques. Techniques for data abstraction include filtering \cite{conf/chi/AhlbergS94b}, clustering
and sampling \cite{553159}. Clutter reduction assigns more space to interesting data and tries to hide less interesting data.
The most common techniques for clutter reduction include zooming and distortion \cite{559215,1382895,1196005}.


We first overview some previoous work on multidimensional and hierarchical data visualizations. 
Shneiderman \cite{545307} proposes a task-by-data-type taxonomy of information visualizations of multi-dimensional
data types as well as structured data which provides researchers and developers with a guideline for
the design and implementation of visualization systems. His work highlights the importance of data
abstraction operations such as summarization, filtering, zooming and extraction.

Jeong and Pang \cite{729555} present a technique called reconfigurable disc trees for visualizing large 
hierarchical data sets. The data is presented as a tree which can be laid out in two or three dimensions.
Their visualization is based on discs around which the children of each node are placed and eliminates
visual overlaps among subtrees.

Johnson and Shneiderman \cite{Johnson:1991:TSA:949607.949654} present another technique for visualizing hierarchical
information called TreeMaps. TreeMaps recursively partition the display into rectangular bounding boxes representing the 
tree structure. The drawing of nodes within their bounding box is dependent on their content and can be interactively
controlled.

Fua et. al. \cite{Fua:2000:SBM:614278.614457} present a mechanism for navigating hierarchically organized structures
called Structure-Based brushes. Brushing consists of painting sections of the display using a mouse or stylus, indicating
data items to be selected. Their technique can be used to perform selection in datasets organized via hierarchical
clustering and partitioning algorithms and serves to perform subset selection for further analysis. The technique defines
the level of detail which can be e.g. the cluster size, volume or the level number which can be used to traverse the
hierarchy.

Elmqvist et. al. \cite{Elmqvist:2010:HAI:1749404.1749525} present a model for building, visualizing and interacting with
multi-scale representations of visual information using hierarchical aggregation. Their model allows for augmenting visualization
techniques into a multi-scale structure using hierarchical aggregation.

In their work, Cui et. al. \cite{4015421} investigate data abstraction quality measures in multi-resolution
visualization systems. Specifically, they propose a histogram difference measure and a Nearest neighbor
measure to measure the quality of the abstraction.

\lstopar{One important distinction between this work and ours is that this methods are not tailored to time-series data.}


Visualization in data mining or/and exploratory data analysis range from simple techniques for data exploration, such as
scatter plots, box plots, heatmaps, etc., to visualization of more complex structures such as \lstopar{dimensionality reduction \cite{Fortuna05visualizationof,Poco:2011:FEM:2421953.2422017,Vesanto99som-baseddata,7192673}},
association rules \cite{Hahsler_visualizingassociation}, decision trees \cite{963292,Nguyen2000}, clusters and dendrograms.

Oliveira and Levkowitz \cite{1207445} provide a survey of visualization tools used for data mining. Their survey focuses
to visualization of tabular data and provides an overview of tools and techniques for data exploration,
visualization of the extracted knowledge, discussing the question of how to select an appropriate visualization
technique.

Keim \cite{981847} provides a classification of information visualization techniques used in data mining, which is based on
the data type, visualization technique and interaction and distortion technique.

Benz et. al. \cite{Benz2004239} present a multi-resolution framework for analysis of image data, by exploring a hierarchical
image object network and representing strongly linked objects using polygons and fuzzy systems.


We base our techniques on  Markov chains. Markov chains are a special kind of memoryless stochastic process and typically only consist of a countable number of states. These two properties make them useful for many applications as they allow the computation of predictions as
well as quantifying their behavior by calculating probabilities and expected values. Markov chains are widely used for modelling systems dating back to Andrey Markov himself \cite{markov13} in 1913. Since then
Markov chains have had many famous applications such as Shannon's application to information theory \cite{Shannon:1948},
Baum's Hidden Markov Models (HMM) \cite{baum1970} and the application to web search by Sergey Brin and Larry Page \cite{Lawrence981}.
%
Other applications include the simulation of protein folding \cite{pande-beauchamp-bowman:2010:methods:markov-model-review},
analysis of biochemical networks \cite{Ciocchetta2009145},
genetics \cite{Huelsenbeck2310}, sensor networks \cite{DBLP:journals/corr/AlsheikhHNTL15}, agriculture, queuing theory, etc.

There has been some work on learning Markov chains. In their paper, Deng et. al. \cite{5746509} explored the aggregation of Markov chains using an information-theoretic
approach. Their work focuses on model reduction of complex discrete time Markov chains, using Kullback-Leibler divergence
rate to measure the difference between the original model and the approximation and provides a recursive bi-partitioning
algorithm for partitioning discrete time Markov chains. Their work provides a state aggregation solution by averaging
transition probabilities based on the stationary distribution. In our work we adapt this solution to the continuous time
setting.

There has also been a large body on work on approximate Markov chains for Monte Carlo Markov Chain (MCMC) simulation \cite{RSSD:RSSD117,HASTINGS01041970,10.2307/2246094,1512059}. In this case, however, the emphasis is on reducing the mixing time to ensure fast convergence. In our case, we want to preserve as much of the dynamics as possible. 


\begin{comment}
Large multivariate datasets have become common in many applications, including production lines,
monitoring systems, bioinformatics and social sciences. As the number of observations increases,
existing tools become cluttered and unresponsive. This clutter saturates visualization and hinders
data analysts as large response times make interactive exploration difficult.

Many techniques have been proposed in the literature that address this scalability problem.
They can generally be classified into two groups: data abstraction and clutter reduction 
techniques.

Techniques for data abstraction include filtering [TODO cite], clustering
and sampling [TODO cite].

Clutter reduction assigns more space to interesting data and tries to hide less interesting data.
The most common techniques for clutter reduction include zooming and distortion.

This section starts by providing the state-of-the-art in clustering and Markov chain research and
then continues with an overview of existing visualization techniques.



\subsection{General Visualization Techniques}


Shneiderman \cite{545307} proposes a task-by-data-type taxonomy of information visualizations of multi-dimensional
data types as well as structured data which provides researchers and developers with a guideline for
the design and implementation of visualization systems. His work highlights the importance of data
abstraction operations such as summarization, filtering, zooming and extraction.

Jeong and Pang \cite{729555} present a technique called reconfigurable disc trees for visualizing large 
hierarchical data sets. The data is presented as a tree which can be laid out in two or three dimensions.
Their visualization is based on discs around which the children of each node are placed and eliminates
visual overlaps among subtrees.

Johnson and Shneiderman \cite{Johnson:1991:TSA:949607.949654} present another technique for visualizing hierarchical
information called TreeMaps. TreeMaps recursively partition the display into rectangular bounding boxes representing the 
tree structure. The drawing of nodes within their bounding box is dependent on their content and can be interactively
controlled.

Fua et. al. \cite{Fua:2000:SBM:614278.614457} present a mechanism for navigating hierarchically organized structures
called Structure-Based brushes. Brushing consists of painting sections of the display using a mouse or stylus, indicating
data items to be selected. Their technique can be used to perform selection in datasets organized via hierarchical
clustering and partitioning algorithms and serves to perform subset selection for further analysis. The technique defines
the level of detail which can be e.g. the cluster size, volume of the level number which can be used to traverse the
hierarchy.

Elmqvist et. al. \cite{Elmqvist:2010:HAI:1749404.1749525} present a model for building, visualizing and interacting with
multi-scale representations of visual information using hierarchical aggregation. Their model allows for augmenting visualization
techniques into a multi-scale structure using hierarchical aggregation.

In their work, Cui et. al. \cite{4015421} investigate data abstraction quality measures in multi-resolution
visualization systems. Specifically, they propose a histogram difference measure and a Nearest neighbor
measure to measure the quality of the abstraction.

\subsection{Visualization in Data Mining}

Visualization is an important tool in data mining as it \textcolor{red}{...}. Data mining is commonly defined
as the extraction of patterns or models from data sets, usually as part of extracting high-level
knowledge from low-level data. Visualization is an important tool in this area as it gives analysts a
tool for exploring, understanding data and creating hypothesis.

Visualization techniques used in data mining range from simple techniques for data exploration, such as
scatter plots, box plots, heatmaps, etc., to visualization of more complex structures such as PCA \lstopar{[TODO cite]},
association rules, decision trees \lstopar{[TODO cite]}, clusters and dendrograms.

Oliveira and Levkowitz \cite{1207445} provide a survey of visualization tools used for data mining. Their survey focuses
to visualization of tabular data and provides an overview of tools and techniques for data exploration,
visualization of the extracted knowledge, discussing the question of how to select an appropriate visualization
technique.

Keim \cite{981847} provides a classification of information visualization techniques used in data mining, which is based on
the data type, visualization technique and interaction and distortion technique.

Benz et. al. \cite{Benz2004239} present a multi-resolution framework for analysis of image data, by exploring a hierarchical
image object network and representing strongly linked objects using polygons and fuzzy systems.

\subsection{Markov Chains}

Markov chains are a special kind of stochastic process
which can assume only a finite or countable number of states and have no memory of where they have been in the past.
These two properties make them useful for many applications as they allow the computation of predictions as
well as quantifying their behavior by calculating probabilities and expected values.

The first application of Markov chains begins with Andrey Markov himself \cite{markov13} in 1913. Since then
Markov chains have had many famous applications such as Shannon's application to information theory \cite{Shannon:1948},
Baum's Hidden Markov Models (HMM) \cite{baum1970} and the application to web search by Sergey Brin and Larry Page \cite{Lawrence981}.

Other applications include the simulation of protein folding \cite{pande-beauchamp-bowman:2010:methods:markov-model-review},
analysis of biochemical networks \cite{Ciocchetta2009145},
genetics \cite{Huelsenbeck2310}, sensor networks \cite{DBLP:journals/corr/AlsheikhHNTL15}, agriculture, queuing theory, etc.

In their paper, Deng et. al. \cite{5746509} explored the aggregation of Markov chains using an information-theoretic
approach. Their work focuses on model reduction of complex discrete time Markov chains, using Kullback-Leibler divergence
rate to measure the difference between the original model and the approximation and provides a recursive bi-partitioning
algorithm for partitioning discrete time Markov chains. Their work provides a state aggregation solution by averaging
transition probabilities based on the stationary distribution. In our work we adapt this solution to the continuous time
setting.

\lstopar{Hierarchical structures are used in many real-world applications, as they are flexible in storing information
and allow the user to zoom into detail high detail levels, while hiding these details on the upper levels.}

\lstopar{We differ from the related work in that we summarize the information, instead of drawing the whole
dataset. We also summarize the dynamics of the dataset.}
\end{comment}

\section{Preliminaries}
\label{sec:preliminaries}
Our framework is aimed at visualizing multivariate time-series. Here, we introduce the tools required in our framework. We assume as input we are given samples of $d$ signals, which we interpret as one multi-dimensional signal:
$$f: x(t)\mapsto \R^d$$
For simplicity, we assume that the signals are continuous and real-valued as well as uniformly sampled. This means that all signals are sampled simultaneously and at constant intervals. These assumptions can be relaxed in practice, with a preprocessing resampling step.  Each sample then consists of a vector of samples from all input signals at time $t$. 
From this point on we  assume that the image of $f$ defines a compact subset of Euclidean space. For our input, we assume a PL-reconstruction based on time-adjacent samples. 
% MOVE TO DISCUSSION: If our system is intrinsically $d$-dimnesional, then by Whitney's embedding theorem\cite{}, if we have $2d+1$ independent signals, we should be able to reconstruct the system.

% MOVE TO EXPERIMENT SECTION: If we do not have enough independent signals, we can always use a time-delay embedding to raise the dimension of our signal using Takens embedding theorem \cite{} and extensions \cite{}.  

% A time-delay embedding is defined as a function:
% $$ f(x(t),d,\tau) = (x(t),x(t+\tau),\ldots,x(t+(d-1)\tau) )$$
% It takes a time series and lifts it to $d$ dimensions by considering samples spaced at $\tau$ intervals. There is considerable amount of work on how to choose $\tau$, but in our case we often (a) have sufficiently many signals (b) almost any value of $\tau$ works. 

Our main modeling tool is the continuous time Markov chain (CTMC). Markov chains are Markov processes with a finite or countable number of states. The main characteristic of a Markov process is that it has no memory, which means that only its current state influences where it goes next \cite{norris1998markov}.

Markov chains are ubiquitous in modeling over a wide range of phenomena and applications. What makes them particularly useful is that the memoryless property makes it computationally efficient to compute probabilities and expected values of future events.

Formally, a Markov chain is a stochastic process $(X_t)_{t \ge 0}$ which can assume 
only a finite or countable number of states $i \in I$. The starting state
is sampled from a probability distribution $\lambda$, called the initial distribution.
We define $p_{ij}(t)$ to be the probability of the process being in state $j$ at time $t$
when starting from state $i$ at time $0$ and a family of stochastic matrices $\{P(t)\}_{t \ge 0}$,
where the $(i,j)$-th element of $P_{ij}(t)$ equals $p_{ij}(t)$.
Each row of $P(t)$ is thus a probability distribution over the state space $I$. In our work we assume $(X_t)_{t \ge 0}$ irreducible, non-explosive and has a finite number of states. \lstopar{we assume that $(X_t)_{t \ge 0}$  is irreducible ... ???}

% \lstopar{
% \begin{itemize}
% 	\item The state space is finite!
% \end{itemize}
%}

The two most common types of Markov chains are discrete time and continuous time.
Discrete time Markov chains are the most basic type of Markov chains, where the process
changes states in discrete time steps $n \in \mathbb{N}$ with probabilities $p_{ij} = p_{ij}(1)$
and obeys the Markov property presented in Definition \ref{thm:markov-property-discrete}.

\begin{defn}
\label{thm:markov-property-discrete}
Let $(X_n)_{n \ge 0}$ be a discrete time Markov chain with initial distribution $\lambda$.
Then conditional on $X_m = i$, $(X_{m + n})_{n \ge 0}$ is also a Markov chain with initial
distribution $\delta_i$ and is independent of $X_0, X_1, ..., X_m$.
\end{defn}

Continuous time Markov chains are a generalization of discrete time Markov chains as they
fill the gap between the discrete time steps and can change states at any given moment.
Continuous time Markov chains are closely related to Poisson processes. State transition \lstopar{transitions???} can be seen as a Poisson arrival process which depends on the current state. 
%
 %Imagine a labyrinth
%of corridors and chambers and associate each chamber with a single state from the state space
%$I$. The corridor between chambers $i$ and $j$
%is shut by a single door that opens for an infinitely small amount of time at the jump times of
%a Poisson process with rate $q_{ij}$. If the person walking through the labyrinth changes
%chambers each time a door opens, they are performing
%a continuous time Markov chain.
%\primoz{the above paragraph should be rewritten or removed}
%
The basic data needed to define a continuous time Markov chain on state space $I$ are
given by a transition rate matrix $Q$ satisfying three conditions:
\begin{enumerate}
	\item $q_{ij} \ge 0$ for all $i \ne j$
	\item $\sum_{j \in I} q_{ij} = 0$
	\item $-\infty < q_{ii} \le 0$ for all $i \in I$.
\end{enumerate}

Each off-diagonal element of $Q$, $q_{ij}$ represents the rate of jumping from state $i$ to state $j$,
while the diagonal elements $q_{ii}$ represent the rate of leaving $i$. The stochastic matrix
$P(t)$ is represented as $P(t) = e^{tQ}$ and can be computed by solving Kolmogorov's equations:
\begin{equation}
	\nonumber
	\frac{d}{dt}P(t) = P(t)Q
\end{equation}
with initial condition $P(0) = I$.
%
Markov chains can be defined in terms of $n(n-1)$ independent Poisson processes
of rates $q_{ij}$. However, the definition below is much more common:%Definition~\ref{def:ctmc} is \lstopar{much more common in the literature}.
\begin{defn}
	\label{def:ctmc}
	Let $(X_t)_{t \ge 0}$ be a right-continuous process assuming values in $I$ and let $Q$
	be a transition rate matrix. Then for all $t, \epsilon \ge 0$, conditional on $X_t = i$,
	$X_{t+\epsilon}$ is independent of $(X_s)_{0 \le s \le t}$ and as $\epsilon \downarrow 0$
	\begin{equation}
		\nonumber
		P(X_{t+\epsilon} = j | X_t = t) = p_{ij}(\epsilon) = \delta_{ij} + q_{ij}\epsilon + o(\epsilon)
	\end{equation}
	where $\delta_{ij}$ is the Dirac delta function \cite{norris1998markov}.
\end{defn}
When working with Markov
chains one is often interested in its long term properties. One such property is the
stationary distribution, $\pi$. One can show, that for irreducible and non-explosive Markov chains,
the stationary distribution always exists~\cite{norris1998markov} and can be computed as the left eigenvector of $Q$ 
corresponding to the eigenvalue $0$:
\begin{equation}
	\nonumber
	\pi Q = 0.
\end{equation}
In our case, the stationary distribution is interesting because by the convergence theorem~\cite{aldous-fill-2014}
each element $\pi_i$ represents the proportion of time the process spends in state $i$:
\begin{equation}
	\nonumber
	\lim\limits_{t \rightarrow \infty} P(X_t = i) = \pi_i, \forall i \in I.
\end{equation}

To go from our piecewise-linear (PL) input signal to a Markov chain, we consider a spatial discretization of the signal.
We define a \emph{partition function} $g:\R^d\rightarrow I$, where $I$ is a finite, discrete set. This set then represents the states of the Markov chain. If we assume our system is measurable on $\R^d$, then $g$ induces a measure on the set $I$, i.e., let $x(t)\in \R^d$ be the state of the system at time $t$. This induces a corresponding  state $a(t) = g(x(t)) \in I$. This function also defines transition probabililties between the different states. Let $a,b\in I$, then the joint distribution is
    $$\mathbb{P}(a(t_1),b(t_2)) = \mathbb{P}\left(x(t_1)\in g^{-1}(a), y(t_2) \in g^{-1}(b)\right), 	 \quad t_1\leq t_2$$
 In general, this is not Markovian for continuous processes\footnote{This is not Markovian, even for Brownian motion if dimension$>1$ }. The Markovian assumption we make in this step represents the fundamental information loss in our approach\footnote{We can construct arbitrarily good approximations, however the processes do not converge in the limit.}.
Formally, this allows us to define a second map from a measure in $\R^d$ to a Markov chain. However, to avoid overcomplicating the notation, we use $g$ for both. 

 A map between partitions $\phi: I_1 \rightarrow I_2$, is well-defined if it maps each cell of $I_1$ to precisely one cell of $I_2$. That is, for all $x\in\R^d$, ${\phi\circ g_1(x) =  g_2(x)}$\lstopar{for some $g_1, g_2$}. If this relationship holds, it follows that this induces a measurable map between $I_1$ and $I_2$. We use this map to construct a map between states of the Markov chains, which we denote $\Phi$.  We note that in our setting, $\Phi$ can be used to define a map between partitions, since there is a one-to-one correspondence between the partition cells and the states of the Markov chain.  
%\primoz{Last sentence, I am not 100 percent sure about. }
% Let $S\subseteq \R^d$, be the support of our input signal. 

% Given a partition of $S$, denoted by  $\cP$, 
% we can construct a Markov chain. Construct a  state in a Markov chain $\cM_i \in \cM$ for each  partition cell $\cP_i$. Let this mapping be denoted by $f:\cP \rightarrow \cM$. We can endow our data with a canonical measure induced from the volume measure on $\R^{d+1}$ \primoz{here we have to be a bit careful about time}. This induces a measure on the partitions $\mu$, which by the  pushforward $f(\mu)$, induces a measure on the states. This represents the empirical stationary distribution on our Markov chain. To define the transition probabilities, we note that it can be defined in terms of the adjacent partition elements. 
% \primoz{the above paragraph should be synchronized with framework and maps....}

% This completely defines a continuous time Markov chain. Here, we would like to note that this step represents the fundamental information loss in our approach. % DISCUSSION: The transition from continuous process to discrete space (without memory) inherently loses information in any dimension larger than 1.  
% %
% The Markov chain we obtain is highly dependent on the choice of partition. In principle, the initial discretization step  results in a  Markov chain with many states corresponding to a fine partition of the space. Our main contribution is to consider the Markov chains over multiple scales. %This corresponds to a \emph{hierarchical partition of space}. From above, we have that any partition induces a Markov chain. 

% If we are given a sequence of increasingly coarse paritions there is a well defined surjective map 
% $\cP(i)\rightarrow\cP(j)$ from finer to coarser. Each such map also induces a well-defined map between Markov chains. \primoz{Can we prove that these are the same as what we do?}

% In practice, we are not given a hierarchical partition of space, but rather using our initial Markov chain, we define a map between Markov chains by merging states. Note that any such sequence does correspond to a hierarchical partition. \primoz{Maybe add a two line proof here}.  In the following section, we describe how we compute the smaller Markov chain given an initial Markov chain and a map. We also describe how we choose which states to merge, but the intuition is to merge states so that as little of the underlying dynamics is lost. 

% In Section~\ref{sec:multiscale-implementation}, we decribe how we constructed the underlying partition and our initial Markov chain.  

% \primoz{This last part still needs work}

% hierarchical partitions of space

% maps between markov chains



\section{Framework}
\label{sec:multiscale-framework}
\iffalse
\begin{itemize}
	\item Initial state construction (State Identification)
	\item Aggregation (State Aggregation)
	\item Transition probabililties (Modeling transitions)
\end{itemize}
\fi

The key idea at the heart of our system is to build and visualize an abstraction of the underlying 
dynamics of the system. Our abstraction is based on Markov chains where we represent the dynamics of the data using states and transitions. A key feature of our system is \lstopar{to} present this abstraction over \emph{multiple scales}. 
%
%To capture and summarize the dynamics of the system our methodology represents the data
%in a qualitative manner using states and transitions. 
%
We begin by presenting an overview
of our multi-scale methodology. The methodology consists of four main steps as shown in Figure \ref{fig:methodology}.
\begin{figure*}[h!]
	\centering
	\includegraphics[width=\textwidth]{overall1-crop}
	\caption{Overview of the proposed three-step methodology.}
	\label{fig:methodology}
\end{figure*}

We begin by considering the multivariate time series as a point cloud in high dimensional space, ignoring the time component. The second step identifies the underlying structure of this point cloud, identifying their most typical states. In Figure~\ref{fig:methodology}, we see two noisy periodic signals which are partitioned roughly according to the phase of the system. As described in Section \lstopar{\ref{} xxx}, this is done by considering a metric on the data points and partitioning or clustering them. Each cluster/partition is then associated to a state in a Markov model. We refer to these as the low level states. The 
 dynamics are modeled through the transition probabilities between the states using continuous time Markov chains (Section \ref{sec:preliminaries}). Finally, we construct a hierarchy of such Markov chains by aggregating the initial states into coarser states. Each level of the hierarchy is associated to a unique scale, yielding a multi-scale model.

%The methodology starts in the top left part of the figure with an initial step that summarizes
%the structure of the data by identifying their most typical states. It achieves this by \lstopar{placing}
%the data in a \lstopar{design time} metric space and partitioning them. It then associates
%each partition with a single state.
% \lstopar{
% \begin{itemize}
% 	\item These states are used as the lowest-level states.
% \end{itemize}
% }
%In the second step we summarize dynamics by modeling transition probabilities using continuous time Markov chain framework presented in Section \ref{sec:preliminaries}.
% The third and final step aggregates states and transitions into a hierarchy, associating each
% level of the hierarchy with a unique scale, thus providing a multi-scale model. These steps 
% are explained in detail in the following subsections.


% \subsection{Initial State Construction}
% \label{sec:framework-states}

To construct the initial low level states, we define a partition function $\lstopar{f}: \mathbb{R}^d \rightarrow I$ mapping the $d$-dimensional signal to a finite, discrete set $I$. These represent the states which represent tha basic unit of abstraction in our visualization. To complete the construction of the continuous-time Markov chain, we must define the transition rates.

%The first step in our methodology is the construction of initial lowest level states. Using the notation
%from Section \ref{sec:preliminaries} we define a partition function 
%from $\lstopar{f}: \mathbb{R}^d \rightarrow I$ mapping the $d$-dimensional signal to lowest level states
% \lstopar{
% with the following properties:
% \begin{itemize}
% 	\item similar points in $\mathbb{R}^d$ should be mapped to the same or neighboring partitions.
% \end{itemize} 
% }

%\subsection{Modeling Transition Probabilities}
%\label{sec:framework-transitions}

%While we represent the states as described in Section \ref{sec:framework-states},
Recall that all the data needed to represent a continuous-time Markov chain
is stored as transition rates in a transition rate matrix $Q \in \mathbb{R}^{n \times n}$, where $n$ represents
the cardinality of the finite state space $I$ (Section \ref{sec:preliminaries}). However, initial user
feedback showed that transition rates are not a very informative way of visualizing state transitions. Instead, 
we choose to visualize transitions in terms of the jump chain $\Pi$.
This represents an alternative, but equivalent, representation of a Markov chain by its jump chain and holding times. 

The jump chain of a continuous time Markov chain $(X_t)_{t \ge 0}$ is
a discrete time Markov chain $(Y_n)_{n \ge 0}$ with transition matrix $\Pi$ defined as
\begin{equation}
	\nonumber
	\left(\Pi\right)_{ij} = 
		\left\{
			\begin{array}{ll}
				-\frac{q_{ij}}{q_{ii}} & \mbox{if } i \ne j, q_{ii} \ne 0 \\
				0 & \mbox{if } i \ne j, q_{ii} = 0
			\end{array}
		\right.
\end{equation}
\begin{equation}
	\nonumber
	\left(\Pi\right)_{ii} = 
		\left\{
			\begin{array}{ll}
				0 & \mbox{if } q_{ii} \ne 0 \\
				1 & \mbox{if } q_{ii} = 0
			\end{array}
		\right.
\end{equation}

The formal definition of a Markov chain in terms of its jump chain and holding times is given belows.
\begin{defn}
	\label{def:jump-chain-holding-times}
	A right-continuous process $(X_t)_{t \ge 0}$ is a continuous time Markov chain with initial
	distribution $\lambda$ and transition rate matrix $Q$ if its jump chain $(Y_n)_{n \ge 0}$ is a 
	discrete time Markov chain with initial distribution $\lambda$ and transition matrix $\Pi$ and
	for each $n \ge 1$, conditional on $Y_0, Y_1, ..., Y_{n-1}$, its holding times $S_1, S_2, ..., S_{n-1}$
	are independent exponentially distributed random variables of parameters $q_{Y_0}, q_{Y_1}, ..., q_{Y_{n-1}}$
	where $q_i = -q_{ii}$.
\end{defn}

The states and jump matrix define a directed graph which we use to visualize the system. The widths of the arrows represent the corresponding values of the jump matrix. This only encodes the state transitions. To visualize how long the system spends in a state, we use the size of the state.
To determine the size of each state, we turn to the ergodic properties of $(X_t)_{t \ge 0}$. Specifically
we use the elements of the stationary distribution $\pi = (\pi_1, \pi_2, ..., \pi_n)$ and draw the area of
state $i$ proportional to $\pi_i$. We note that under the assumptions presented in Section \ref{sec:preliminaries}
this distribution always exists. \primoz{why do we not use the holding times  - a sentence or two here}

\subsection{Aggregation}
\label{sec:framework-aggregation}

One of the main contributions of this paper is to visualize the system at multiple scales  representing
the data at different detail levels - from the finest to the coarsest. To build our multi-scale representation we take as input, the continuous time Markov chain, we initially constructed. 

Recall from Section \ref{sec:preliminaries}, rather than constructing a single partition of our space, we can construct a hierarchical partition. With this hierarchical partition,  each scale $s$ is associated with a specific partition
function $\lstopar{f_s}: \mathbb{R}^d \rightarrow \lstopar{I_s}$.

At the finest scale, we use the 
partition function \lstopar{$f: \mathbb{R}^d \rightarrow I$}, which induces
the state space $I$ of the Markov chain $(X_t)_{t \ge 0}$. We then compute the partition
function for coarser scales by aggregating elements of $I$, inducing state space $I_s$.
Thus $I_s$ represents a partition of $I$ and can again be used as the state space on scale $s$.
$I_s$ is determined by a map $\phi_s: I \rightarrow I_s$. This
way $f_s$, can be represented as \lstopar{$f_s = f \circ \phi_s$}.

The details of how compute these maps and partitions is described in Section~\ref{}. Therefore, the problem is given $I$, $I_s$, $f$ and the Markov chain induced by $I$, to compute the Markov chain induced by $I_s$.

The states of this Markov chain $(X_t^{(s)})_{t \ge 0}$ are induced by $I_s$, what remains is to compute the corresponding  transition
rate matrix $Q^{(s)}$. To compute $Q^{(s)}$, we adapt the formula
proposed in \cite{5746509} to continuous time Markov chains. We define a partition function
$\Phi: \mathbb{R}^{n \times n} \rightarrow \mathbb{R}^{m \times m}$ with $m \le n$ using formula~\ref{eq:ctmc-state-aggregation}.
\begin{equation}
	\label{eq:ctmc-state-aggregation}
	\Phi(Q) = (P' \Pi P)^{-1} P' \Pi Q P
\end{equation}
where $\Pi = diag(\pi)$, $\pi$ is the stationary distribution of $(X_t)_{t \ge 0}$ and $P$ is a 
$n \times m$ matrix with elements
\begin{equation}
	\nonumber
	\left(P\right)_{ij} = 
		\left\{
			\begin{array}{ll}
				1 & \mbox{if } \phi(i) = j \\
				0 & \mbox{otherwise}.
			\end{array}
		\right.
\end{equation}
Thus the data needed to represent a Markov chain at a specific scale $s$ is obtained as $Q^{(s)} = \Phi(Q)$.
If we define $\psi = \phi \circ \phi^{-1}$, then Equation \ref{eq:ctmc-state-aggregation} can be rewritten as
\begin{equation}
	\nonumber
	q_{\phi(i)\phi(j)}^{(s)} = \frac{\sum\limits_{i \in \psi(i)}\pi_i \sum\limits_{j \in \psi(j)} q_{ij}}{\sum\limits_{i \in \psi(i)}\pi_i}
\end{equation}
$\pi^{(s)}$, the stationary distribution of $(X_t^{(s)})_{t \ge 0}$, can be computed directly from
the stationary distribution of the original chain $(X_t)_{t \ge 0}$ by the following rule:
\begin{equation}
	\nonumber
	\pi^{(s)} = \pi P.
\end{equation}
Intuitively, the stationary distributions are preserved through the scales, although the intermediate dynamics may change. How much, depends on how we construct our hierarchical partition function. 



\section{Realization}
\label{sec:multiscale-implementation}
\begin{itemize}
	\item clustering
	
	The clustering step of our multi-level framework is used to identify typical low-level states
	of the visualized system from the raw observations. It achieves this by partitioning the
	observations and associating each state with a single partition.
	
	The literature suggests many partitioning algorithms, including those that produce a hierarchical 
	partition. The computational complexity of these algorithms however is $O(n^2)$ which renders 
	them unsuitable for large datasets. We therefore first create a flat partition of the data space
	using K-Means, resulting in a Voronoi diagram and enabling us to construct states by assigning
	observations to the partition with the nearest centroid.
	
	\item criteria for aggregation 
	
	Once the lowest level states are constructed, the next step is to aggregate them to obtain a
	multi-resolution view of the data. This process results in a hierarchical tree structure
	which essentially induces a hierarchical partition on the data space, with partitions
	on the higher levels of the tree structure containing partitions on the lower levels of the
	tree structure.
	
	 A common approach when performing aggregation is to aggregate
	items that are similar to each other in some way.
	
	We choose a bottom-up hierarchical clustering algorithm,
	where we aggregate two states if they lie close to each other in Euclidean space.
	
	\item modeling transition rates
	
	We model state transitions using a continuous time Markov chain framework presented in Section
	\ref{sec:preliminaries}. The transitions are modeled on the lowest level states.
	
	We allow users to select a subset of the attributes which we use to model transition rates.
	Since the jump process from state $i$ to state $j$ can be characterized as a Poisson process,
	we can model its transition rate $q_{ij}$ as a function of these attributes $q_{ij}(x_i)$.
	We do this by first discretizing the continuous time parameter into a discrete sequence
	$(0, \epsilon, 2\epsilon, ...)$ and estimating the transition rates as
	\begin{equation}
		q_{ij} = \frac{\epsilon}{\tilde{p}_{ij}}
	\end{equation}
	where $\tilde{p}_{ij}$ is the estimated probability of jumping from state $i$ to state $j$
	in time $\epsilon$.
	
	Suppose the process is in state $i$ at time $t$ and define a random variable 
	$J_i = j \Leftrightarrow X_{t + \epsilon} = j$. $J_i$ then has a multinomial distribution
	with parameters $(p_{i1}, p_{i1}, ..., p_{in})$ which can be modeled using a nominal
	logistic regression model \cite{glm-introduction} to estimate $\tilde{p}_{ij}$.
	
	The transition rate matrix is then computed on-the-fly from the matrix of logistic regression models.
	
	To model the Markov chain on a specific detail level, we adapt a formula proposed in \cite{5746509}
	to the continuous time framework. We define an aggregated states $\{S_i\}_i$ as non-intersecting
	subsets of the state space $I$ where $\bigcup_i S_i = I$. The transition rate among aggregated
	states can then be computed using formula \ref{eq:agg-transitions}.
	\begin{equation}
		\label{eq:agg-transitions}
		q_{S_i S_j} = \frac{\sum\limits_{i \in S_i}\pi_i \sum\limits_{j \in S_j} q_{ij}}{\sum\limits_{i \in S_i}\pi_i}
	\end{equation}
	where $\pi = (\pi_1, \pi_2, ..., \pi_n)$ represents the stationary distribution of the process. The
	above formula can be rewritten in matrix from as:
	\begin{equation}
		Q_l(Q) = \left(P\Pi P\right)^{-1}P\Pi Q P
	\end{equation}
	where $P$ is a projection matrix defined as
	\begin{equation}
		\left(P\right)_{ij} = 
			\left\{
				\begin{array}{ll}
					1 & \mbox{if } j \in S_i \\
					0 & \mbox{otherwise}.
				\end{array}
			\right.
	\end{equation}

	
\end{itemize}

Be begin this section with an overview of our multi-scale methodology shown in Figure \ref{fig:methodology}.
The methodology starts by aggregating and resampling the data streams, interpolating wherever needed. This
first step is critical for further processing as it produces feature vectors, used in later steps of the
methodology. Our next step includes identifying typical low-level states from the feature vectors
created in step one. We achieve this by partitioning the feature vectors and associating each state
with a single partition. While the literature suggest many hierarchical partitioning methods, including
those that produce a hierarchical partition, their computational complexity is $O(n^2)$ which
renders them unsuitable for big data scenarios. We therefore propose a flat partition of the data space,
constructing the hierarchy later in the process.

The next step includes modeling transitions. Our methodology models transitions using a Markovian model
presented in Section \ref{sec:preliminaries}. The main characteristic of Markov models is that they
retain no memory of where the process has been in the past. This means that only the current state
influences where the process will go next. In this work we are only interested in processes that
can assume a finite set of states, called Markov chains \cite{norris1998markov}.

Hierarchical clustering is based on iteratively building a tree structure of aggregate items
either using a bottom up or top-down approach. Top-down aggregation starts with one aggregate containing
all the items and recursively splitting them until a specific level of granularity has been achieved or
all items belong to their own aggregate.
On the other hand bottom-up aggregation starts by treating each item as its own aggregate and iteratively
merging them until only one aggregate remains.



\section{User Interface}
\label{sec:ui}

\iffalse
Features  
\begin{itemize}
	\item Qualitative representation - states and transitions
	\item State identification services:
	\begin{itemize}
		\item State details and attribute highlighting - histograms + attribute colors
		\item \lstopar{Timeline + parallel coordinates \cite{parcoords} - when do states occur in time}
		\item \lstopar{Coloring states based on attributes}
		\item Decision trees + rule extraction - Explanation of states
		\item Automatic name generation
		\item Zooming into a state + showing paths from a state
	\end{itemize}
\end{itemize}
\fi

The system supports multiple users with datasets and precomputed models (as well as the ability to store new models in the system). Data is generally uploaded via CSV files.  Configuration consists of choice of desired attributes,  clustering method, aggregation strategy and attributes used to model transitions. The mdoel is then constructed and the user can begin interactive exploration.  The construction time varies depending on the size of 
their dataset and configuration. We conducted an experiment to test the performance of our \lstopar{methodology},
presented in section \ref{sec:implementation}.

The interface displays the model using a three panel user interface, an example of
which is shown in Figure \ref{fig:teaser} \lstopar{[TODO teaser reference not appearing]}. The central panel visualizes the model at the current scale as a graph. A state is represented by a circle  whose size represents the time/probability spent in the state, which is computed from the stationary distribution of the Markov chain. The trasnistions are represented by arrows, where again size represents the empirical transition probability. 

The computed model may be used as a monitoring tool, mapping incoming data onto the model in real-time. 
In this mode, the current state (green) and the most likely future states (blue)
are highlighted. The current state is determined by assigning a sample to its nearest 
centroid, as presented in Section \ref{sec:multiscale-implementation}, and is updated in real-time as
data arrives into the system.

The model is initially presented on a coarse scale with only a handful
of states. Users can then explore the model by either traversing the scales using the zoom 
function, as well as isolate and explore individual states using the ``Zoom Into'' function, \lstopar{or explore the 
graph as a tree using the ``Show Path'' function}.

Visualizing a Markov chain as a graph may not be, in itself,  sufficiently informative. The system offers several 
visualizations which assist users in identifying the meaning of states. These range from automatic
state naming, state histograms, attribute highlighting and timeline histograms to decision trees
which visualize the states' properties in a hierarchical manner and \lstopar{extracted rules}.
These services are shown when the user clicks on and selects a state. We describe these visualizations below.  

%and each transition with an arrow on the 
%central panel. The size of a state is calculated from the associated entry in the stationary distribution
%and is proportional to the fraction of time the system spends in the state. The thickness of each arrow
%is proportional to the associated transition probability. 

% The computed 
% When using the model as a
% monitoring tool, 
%After registering, the user is presented with a dashboard, where they can organize their models.
%If they wish to create a new model, they have to upload a CSV file with the dataset they would
%like to visualize. 
%They are then taken through a configuration form, where they select attributes,
%clustering method, aggregation strategy and attributes used to model transitions. When completing
%the form, their new model is constructed. 

%Once constructed, a StreamStory 

\subsection{Probability Distributions and Attribute Highlighting}

The most basic exploratory tool to help a user is to plot the distribution of data inside each state in the form of a histogram. When clicking on a state, the histograms of all the attributes are shows in the right-side panel like in
Figure \ref{fig:teaser} \lstopar{[teaser not showing!]}. Context for each distribution by showing the global distribution of that attribute in the background.

%
%To assist the users to distinguish between states, we highlight each attribute either green of red.
%The color is chosen based on how typical the attribute is for the state. 
To help users understand the  values of attributes within a state, the attributes are colored according to their typical values.  For example, a bright green
value indicates that the attribute typically  has a higher value in this state compared to other states, while
a bright red color suggests, this value is low compared to other states. An example of attribute 
highlighting can be seen in Figure \ref{fig:example-naming-histogram}.

The color is calculated in the initialization step by first classifying all the samples assigned to this state 
against samples assigned to other states using a logistic regression model. \lstopar{The features used in the 
classification are all the dimensions of the input signal, which correspond to all possible attributes.}
The final color is determined by extracting weights from the models and associating positive weights
with green and negative with red.

The users can also view the distribution of attributes across states, by selecting a specific attribute.
When doing so, the states are recolored based on the value of the selected attribute - again, green indicates
a high value, while red indicates a low value.

%%here

\subsection{Automatic State Labeling}

Looking at graphs by itself is not very informative. Although representing the system as states and 
transitions gives insight into the dynamics of the system, it fails to provide a comprehensive summary 
of the dataset. \lstopar{Indeed, our experiments showed first time users were quite confused when viewing
our model with unlabeled states and had difficulty executing even the simplest tasks like finding
states with high temperature in example \ref{sec:experiments-weather}}.

To overcome this issue, an automatic, data-driven, naming service was implemented. The service labels
states based on the inner-state attribute distributions by concatenating the most outlying attribute
with a discrete level, ranging from: lowest, low to high and highest.

Both the attribute and the level used in the label are computed by comparing the inner-state attribute
distributions to the global attribute distributions. The attribute selected by the naming process is
the one with the lowest $p$-value of the inner-cluster $40$-th and $60$-th percentile. The level is
then determined according to the $p$-value. If the $p$-value is below $0.12$ the attribute is labeled
as lowest or highest, while is the $p$-value is below $0.25$ it is labeled low or high. If none of the
attributes achieve a $p$-value of $0.25$ the state is labeled as a mean state. An example of an automatic
label, along with the associated inner-state and global histogram is shown in Figure \ref{fig:example-naming}.

\begin{figure}[h!]
	\centering
	\begin{subfigure}{.48\columnwidth}
	  	\centering
	  	\includegraphics[width=0.7\columnwidth]{example-state-naming}
  		\caption{\label{fig:example-naming-label}\lstopar{TODO}}
	\end{subfigure}
	\begin{subfigure}{.48\columnwidth}
	  	\centering
	  	\includegraphics[width=\columnwidth]{example-state-naming-histogram}
	  	\caption{\label{fig:example-naming-histogram}\lstopar{TODO}}
	\end{subfigure}
	\caption{\lstopar{[TODO]}}
	\label{fig:example-naming}
\end{figure}

\iffalse
In order to assist the user in identifying the meaning of states, the system provides automatic default
state names, based on the distribution of attributes in the state. Each state is given a default name
by combining its most outstanding attribute with a discrete level: LOWEST, LOW, HIGH or
HIGHEST.

The attribute and the level are chosen by comparing its distribution inside the state to the global
distribution in all the states through histograms. This is achieved by first computing the percentiles
of the global distribution. The $40^{th}$ percentile is then computed for the state distribution and
compared against the global distribution. If this percentile lies below the $25^{th}$ or $12^{th}$
percentile, the state is marked with LOW or LOWEST respectively. The final name is chosen according
to the attribute which lies in the lowest percentile.
\fi

\subsection{Decision Trees and Rule Extraction}

An alternate description of a state is generated through the induction of decision trees \cite{Witten:2005:DMP:1205860}. Decision
trees are classification models often used in domains such as medicine for their explanatory power.
When a decision tree is induced, a splitting attribute and cut value are chosen recursively by a
design time criteria. The user can then interpret the tree by traversing the path from the root 
to one of the leafs.

In our use case, we use decision trees as a tool for explaining states. We induce one tree for each
state by classifying the observations of the state against the observations of all other states,
obtaining a quantitative description of the state. We then extract rules from the decision tree,
providing a short summary of the state in the form of $A_i > t_i \cup A_j \in (t_{j_1}, t_{j_2}]$.

Figure \ref{fig:example-decision-tree-and-rule} show an example decision tree and extracted rules
from a state in the weather dataset presented in section \ref{sec:experiments-weather}.

\begin{figure}[h!]
	\centering
	\begin{subfigure}{.3\columnwidth}
	  	\centering
	  	\includegraphics[width=\columnwidth]{example-rule}
  		\caption{\lstopar{TODO}}
  		\label{fig:example-rule}
	\end{subfigure}
	\begin{subfigure}{.68\columnwidth}
	  	\centering
	  	\includegraphics[width=\columnwidth]{example-decision-tree}
	  	\caption{\lstopar{TODO}}
	  	\label{fig:example-decision-tree}
	\end{subfigure}
	\caption{\lstopar{[TODO]}}
	\label{fig:example-decision-tree-and-rule}
\end{figure}

\iffalse
\subsection{Visual Assistance}

When a state becomes selected, the user interface presents the user with several visual aids which
assist them in identifying the states' meaning. The first of these aids is the timeline histogram
which shows the distribution of the states occurrence over time. An example is shown in Figure 
\ref{fig:time-hist}.

\begin{figure}[h!]
	\centering
	\includegraphics[width=\columnwidth]{timeline}
	\caption{[TODO example decision tree].}
	\label{fig:time-hist}
\end{figure}

[TODO histograms]
\fi


\section{Implementation}
\label{sec:implementation}
StreamStory is implemented as part of the QMiner \cite{qminer} data analytics platform,
under the BSD license, with its core functionality written in C++ and exposed as a 
JavaScript API to the Node.js platform, which acts as glue and exposes the functionality
via RESTful API.

\section{Experiments}
\label{sec:experiments}
The system has been used to identify certain patterns in different types of multivariate time-series. In this section, we highlight several examples - the first from synthetically generated data followed by two real world datasets. Finally we describe some other examples of the use of system.

\subsection{Prediction Evaluation}

This first experiment is intended to test the validity of our model. The model was constructed on simulated data of 
an electric motor and is shown in Figure \ref{fig:example-motor}. 
\begin{figure*}[]
  	\centering
  	\begin{subfigure}{.48\textwidth}
	  	\centering
	  	\includegraphics[width=\columnwidth]{simulation-processed}
  		\caption{\label{fig:simulation-chart}\lstopar{TODO}}
	\end{subfigure}
  	\begin{subfigure}{.48\textwidth}
	  	\centering
	  	\includegraphics[width=\columnwidth]{model-motor-simulated}
  		\caption{\label{fig:simulation-model}\lstopar{TODO}}
	\end{subfigure}
  	\caption{\lstopar{TODO}}
  	\label{fig:example-motor}
\end{figure*}
The simulation starts in the leftmost state of Figure~\ref{fig:simulation-model} with the motor in a stationary state
and the power switch turned off. Then \lstopar{an invisible user} randomly, sampled from an exponential distribution,
toggles the ``on" switch. Once the switch is on, as the rotation increases, the model starts moving in the counter clockwise
direction towards the large green state on the right. This state represents the \lstopar{equilibrium} when the temperature
gained through friction equals the temperature lost to the ambient and the \lstopar{signals} become \lstopar{stationary}.
Once the power switch is toggled again, the rotation slowly halts due to friction and the process goes from right to
left in the counter clockwise direction.

To conduct the experiment we generated two dataset: a training set and a test set. Both datasets
contained $200k$ observations. We built a StreamStory model with 20 lowest-level states on the training set using two attributes:
angular velocity and temperature. The training dataset was then replayed through the model and the finest scale states were stored. We then used
the stored states to calculate the transition probabilities and compared them to the models' jump chain $\Pi$. Since a lot of these probabilities were zero, we decided to use only the probabilities that
are non-zero either in the jump chain or in the probabilities calculated from the history. We then
computed the mean absolute error of the non-zero probabilities which resulted in $MAE=0.05$ or $5\%$.

In another experiment we tested the models predictive power. Two StreamStory models were built: (a) one 
model using the same attributes as in the first experiment, while in the second model (b), we used
the logical switch signal to model state transitions. The models were trained on the same dataset
as in the first experiment. Before the process jumped, we extracted the next state probabilities
and used the state with the highest probability as the predicted next state. We then computed
the prediction accuracy as the ratio between the number of correct predictions and the total number
of jumps (total number of predictions). In this experiment model (a) scored $0.845$ while model
(b) scored $0.904$.


\subsection{Weather Data}
\label{sec:experiments-weather}

The example below shows our model generated on monthly rainfall and temperature data
collected over the course of 20 years between 1920 and 1940 in Nottingham, England.

\begin{figure}[h!]
	\centering
	\includegraphics[width=\columnwidth]{example-weather}
	\caption{Qualitative representation of temperature and rainfall data collected over the course of 20 years.}
	\label{fig:example-weather}
\end{figure}

The model was generated using the raw rainfall and temperature data, lifted into a four dimensional space.
\lstopar{Lifting was done as suggested in Section \ref{sec:preliminaries} by appending the previous sample to the current sample, effectively adding two new dimensions}.

The states on the right hand side represent the 
summer states, while the states on the left represent winter states. The yearly timeline flows in the
counter clockwise direction with the spring states residing on the bottom of the figure and the autumn
states on the top.

Interestingly, in this dataset, the rainfall and temperature are very correlated and the auto labeling feature
choose high rainfall as the most significant feature of the summer states. This correlation can be seen
from the attribute histogram shown in Figure \ref{fig:histograms-summer}.

\begin{figure}[h!]
	\centering
	\includegraphics[width=0.5\columnwidth]{histograms-summer}
	\caption{\lstopar{TODO caption}}
	\label{fig:histograms-summer}
\end{figure}

\subsection{GPS Data}

The second example was created using raw GPS coordinates collected using a smartphone between years \lstopar{X} and \lstopar{Y}.
The data represents the everyday movement of a European computer science researcher. Figure \ref{fig:example-geo}
shows our qualitative representation of this data on a high level with 8 states.

\begin{figure}[h!]
	\centering
	\includegraphics[width=\columnwidth]{geo-states}
	\caption{\lstopar{TODO caption}}
	\label{fig:example-geo}
\end{figure}

From the figure, we can see that the system was able to identify the most typical locations of this persons
movements.

They spent most of the time in south-central Europe including the central green state and the bottom-most state. The small state on the right represents India, where they went for vacation last year. On the European continent, the system
also identified Germany and Sweden where the person frequently attends meetings or stays for a short while
during a connecting flight. The states on the left represent the USA with the largest state representing New
York city where the person spent the 2014 summer and the smaller states representing San Francisco and Austin,
Texas.

\subsection{Traffic Data}

Our next example, shown in Figure \ref{fig:example-traffic}, shows a representation of a traffic counter positioned on the highway ring around Ljubljana, Slovenia.
\lstopar{In this example, we lifted the counter into a three dimensional space, by also including the number of
cars three hours and one day before.}
\begin{figure}[h!]
	\centering
	\includegraphics[width=\columnwidth]{example-traffic}
	\caption{\lstopar{TODO caption}}
	\label{fig:example-traffic}
\end{figure}
On a high level, our system was able to identify what we believe is the typical daily cycle on the highway ring.
The state on the left hand side represents a night state, when the traffic traffic is the lowest. This state
lasts from 10 PM to 5 AM when most people are asleep and not much is happening on the road. Then, with a high 
probability, the system jumps into the topmost state with an above average traffic count which lasts from 6 AM
to roughly 9 AM. The next state is the midday state on the right with the highest traffic count lasting from 9 AM
to 7 PM. This is exactly the time when the traffic is very dense and most congestions occur. The next transition
is to the bottommost ``evening" state with an average traffic count between 300 and 1000 cars per \lstopar{hour},
lasting from 8 PM to approximately 10 PM. 

\primoz{
\begin{itemize}
\item Here we also need some examples of where its not useful -  a few different scales and we need references on the data. These can be deliverables. 
\item Also we should show how much information we lose over different scales.
\end{itemize}}

\subsection{Domain Experts}
In the above examples, the discovered states could be interpreted by non-experts. We have also tested the system in more specialized settings. One example the system was tested on measurements from an oil drilling platform\footnote{Unfortunately, at the time of submission, we were not cleared to share the results of the analysis, including an evaluation on useability.}. In this data, natural recurrent behaviour was detected.  The model also detected an ``event,'' where certain equipment failed and drilling needed to be halted.  

%A second example is based on a manufacturing plant. In this case, we cannot 



\section{Discussion}
\label{sec:discussion}
\begin{itemize}
	\item Kaj se zgodi, ce model updates v real time?
	\item Automatic description generation.
\end{itemize}

%% if specified like this the section will be committed in review mode
% \acknowledgments{
% The authors would like to thank 

% The GPS data used in this paper were collected by Luka Bradesko, Jozef Stefan Institute.
% The research leading to these results has received funding from the European Union's Seventh Framework Programme (FP7/2007-2013) under grant agreement n$^{\circ}$ 612329 - ProaSense.
% \lstopar{Reflects only the author’s view and that the Union is not liable for any use that may be made of the
% information contained therein.}

% \lstopar{
% The authors wish to thank A, B, C.
% ProaSense:
% \begin{itemize}
% 	\item Grant number: 612329
% 	\item Call (part) identifier: FP7-ICT-2013-10
% \end{itemize}
% }

% }



\bibliographystyle{abbrv}
%%use following if all content of bibtex file should be shown
%\nocite{*}
\bibliography{streamstory.bib}
\end{document}

