\documentclass[journal]{vgtc}                % final (journal style)
%\documentclass[review,journal]{vgtc}         % review (journal style)
%\documentclass[widereview]{vgtc}             % wide-spaced review
%\documentclass[preprint,journal]{vgtc}       % preprint (journal style)
%\documentclass[electronic,journal]{vgtc}     % electronic version, journal

%% Uncomment one of the lines above depending on where your paper is
%% in the conference process. ``review'' and ``widereview'' are for review
%% submission, ``preprint'' is for pre-publication, and the final version
%% doesn't use a specific qualifier. Further, ``electronic'' includes
%% hyperreferences for more convenient online viewing.

%% Please use one of the ``review'' options in combination with the
%% assigned online id (see below) ONLY if your paper uses a double blind
%% review process. Some conferences, like IEEE Vis and InfoVis, have NOT
%% in the past.

%% Please note that the use of figures other than the optional teaser is not permitted on the first page
%% of the journal version.  Figures should begin on the second page and be
%% in CMYK or Grey scale format, otherwise, colour shifting may occur
%% during the printing process.  Papers submitted with figures other than the optional teaser on the
%% first page will be refused.

%% These three lines bring in essential packages: ``mathptmx'' for Type 1
%% typefaces, ``graphicx'' for inclusion of EPS figures. and ``times''
%% for proper handling of the times font family.

\usepackage{mathptmx}
\usepackage{graphicx}
\usepackage{times}

%% We encourage the use of mathptmx for consistent usage of times font
%% throughout the proceedings. However, if you encounter conflicts
%% with other math-related packages, you may want to disable it.

%% This turns references into clickable hyperlinks.
\usepackage[bookmarks,backref=true,linkcolor=black]{hyperref} %,colorlinks
\hypersetup{
  pdfauthor = {},
  pdftitle = {},
  pdfsubject = {},
  pdfkeywords = {},
  colorlinks=true,
  linkcolor= black,
  citecolor= black,
  pageanchor=true,
  urlcolor = black,
  plainpages = false,
  linktocpage
}

%% If you are submitting a paper to a conference for review with a double
%% blind reviewing process, please replace the value ``0'' below with your
%% OnlineID. Otherwise, you may safely leave it at ``0''.
\onlineid{0}

%% declare the category of your paper, only shown in review mode
\vgtccategory{Research}

%% allow for this line if you want the electronic option to work properly
\vgtcinsertpkg

%% In preprint mode you may define your own headline.
%\preprinttext{To appear in an IEEE VGTC sponsored conference.}

%% Paper title.

\title{A Multi-Level Methodology for Explaining Data Streams}

%% This is how authors are specified in the journal style

%% indicate IEEE Member or Student Member in form indicated below
\author{Luka Stopar and Primoz Skraba}
\authorfooter{
%% insert punctuation at end of each item
\item
Luka Stopar is with Jozef Stefan Institute. E-mail: luka.stopar@ijs.si.
\item
Primoz Skraba is with Jozef Stefan Institute. E-mail: pri0moz.skraba@ijs.si.
}

%other entries to be set up for journal
\shortauthortitle{Stopar \MakeLowercase{\textit{et al.}}: A Multi-Level Methodology for Explaining Data Streams}
%\shortauthortitle{Firstauthor \MakeLowercase{\textit{et al.}}: Paper Title}

%% Abstract section.
\abstract{This paper presents a novel multi-scale methodology for modeling a collection of continuously time-varying
data streams. The data streams are aggregated using unsupervised data mining methods. Typical system states
are then computed on the aggregated data. This is used as input to construct a Markovian transition model
capturing the dynamics of the monitored system. The hierarchical organization of the states enables a
visual representation of the dynamics on multiple aggregation levels.} % end of abstract

%% Keywords that describe your work. Will show as 'Index Terms' in journal
%% please capitalize first letter and insert punctuation after last keyword
\keywords{Radiosity, global illumination, constant time}

%% ACM Computing Classification System (CCS). 
%% See <http://www.acm.org/class/1998/> for details.
%% The ``\CCScat'' command takes four arguments.

\CCScatlist{ % not used in journal version
 \CCScat{K.6.1}{Management of Computing and Information Systems}%
{Project and People Management}{Life Cycle};
 \CCScat{K.7.m}{The Computing Profession}{Miscellaneous}{Ethics}
}


%% UNDO!!!
%% Uncomment below to include a teaser figure.
%  \teaser{
% \centering
% \includegraphics[width=16cm]{}
%  \caption{}
%  }

%% Uncomment below to disable the manuscript note
%\renewcommand{\manuscriptnotetxt}{}

%% Copyright space is enabled by default as required by guidelines.
%% It is disabled by the 'review' option or via the following command:
% \nocopyrightspace

%%%%%%%%%%%%%%%%%%%%%%%%%%%%%%%%%%%%%%%%%%%%%%%%%%%%%%%%%%%%%%%%
%%%%%%%%%%%%%%%%%%%%%% START OF THE PAPER %%%%%%%%%%%%%%%%%%%%%%
%%%%%%%%%%%%%%%%%%%%%%%%%%%%%%%%%%%%%%%%%%%%%%%%%%%%%%%%%%%%%%%%%

\begin{document}

%% The ``\maketitle'' command must be the first command after the
%% ``\begin{document}'' command. It prepares and prints the title block.

\maketitle

%% the only exception to this rule is the \firstsection command
\section{Introduction}
\label{sec:introduction}

Sensory systems typically operate in cycles with a continuously time-varying. These 
systems include, for example, the solar system, manufacturing systems or weather 
systems. Such systems can be characterized by a set of states, along with associated
state transitions. States, on a high level, may include a "day" state and a "night"
state or maybe states with high and low productivity. For example, when a pilot wishes
to change an aircrafts heading, they will put the aircraft into state "banking turn"
by lowering one aileron and raising the other, causing the aircraft to perform a
circular arc. After some time, the wings of the aircraft will be brought level by
an opposing motion of the ailerons and the aircraft will go back into state "level".

Such high-level states can be decomposed into lower-level states, giving us a 
multi-level view of the system and allowing us to observe the system on multiple 
detail levels. For example, a "banking turn" state can be decomposed by the
aircrafts roll and angular velocity, resulting in perhaps three states: "initiate
turn", "full turn" and "end turn".

We present a methodology for modeling such systems and demonstrate its implementation
called StreamStory. StreamStory models such systems as a hierarchical Markovian process
by automatically learning the typical low-level states, transitions and aggregating 
them into a hierarchy, obtaining a unique Markovian process on multiple detail levels.
As such it gives users a unique view into the monitored system.

Furthermore, we divide the inputs streams into two sets: observation and control set.
Attributes in the observation set are the attributes that tell us the state of the 
system and, we assume, cannot directly influence its dynamics. These are parameters
that users cannot directly manipulate, like aircraft tilt from the previous example,
which must be indirectly manipulated through the angles of ailerons. We use observation
attributes to identify, and aggregate, low-level states, detect outliers (anomalies) 
and determine the current state of the system.

In contrast, users can directly manipulate attributes in the control set. These 
are attributes like angles of ailerons that may directly influence the behavior 
(observation attributes) and performance of the system. For example, when an operator 
in a steel factory sets the cooling temperature to a high value, the product will
take longer to go from state "hot" to state "cool". As such, we assume, control attributes
may influence the occurrence, and expected time, of undesired states, associated with
undesired events.

Our system uses control attributes to model state transitions, allowing us to observe
the dynamics with respect to the current configuration and gives us insight into the
expected dynamics with respect to some alternate attribute configuration.

The main research contributions presented in this paper can be summarized as follows:
\begin{enumerate}{}
  \item We present a novel methodology for modeling multivariate continuously time-varying data streams
  in a hierarchical manner providing a unique model of the several detail levels.
  \item We present a novel approach for modeling state transitions in a Markov chain allowing
  to observe the dynamics of the chain under alternate configurations.
  \item We propose a recursive algorithm for partitioning recurrent continuous time Markov chains.
\end{enumerate}


The remainder of this paper is structured as follows. In Section \ref{sec:methodology} we present
our methodology in detail. Section \ref{sec:ui} we present user interaction. Section \ref{sec:experiment}
we present our experimental results and finally in Section \ref{sec:conclusion} we conclude this
paper and give some ideas for further work.


%% \section{Introduction} %for journal use above \firstsection{..} instead


\section{Previous Work}

\section{Preliminaries}
\begin{itemize}
\item Model of data: paths in Euclidean page
\item Markov chains
\end{itemize}

\section{Multiscale Hierarchies - Framework}
\begin{itemize}
\item Initial state construction 
\item Aggregation 
\item Transition probabililties
\end{itemize}


\section{Multiscale Hierarchies - Implementation}
\begin{itemize}
\item clustering
\item criteria for aggregation 
\end{itemize}

\section{User Interface}
Features  

\section{Implementation}

\section{Experiments}
\subsection{Prediction Evaluation}

\subsection{Weather Data}
\subsection{Domain Experts}


\section{Discussion}

%% if specified like this the section will be committed in review mode
\acknowledgments{
The authors wish to thank A, B, C. }

\bibliographystyle{abbrv}
%%use following if all content of bibtex file should be shown
%\nocite{*}
\bibliography{template}
\end{document}

