\label{sec:preliminaries}
Our framework is aimed at visualizing multivariate time-series. Here we introduce the tools required in our framework. We assume as input we are given samples of $d$ signals, which we interpret as one multi-dimensional signal:
$$f: x(t)\mapsto \mathbb{R}^d$$
For simplicity, we assume that the signals are continuous and real-valued as well as uniformly sampled. This means that all signals are sampled simultaneously and at constant intervals. These assumptions can be relaxed in practice, with a preprocessing resampling step.  Each sample then consists of a vector of samples from all input signals at time $t$. 

Assuming we have enough independent signals, we can always use a time-delay embedding to raise the dimension of our signal. 

Further, we assume that the image of $f$ defines a compact subset of Euclidean space. For our input we assume a PL-reconstruction based on time-adjacent samples.

A partition of space 

Markov chain

hierarchical partitions of space

maps between markov chains


\begin{itemize}
\item Model of data: paths in Euclidean page
	\item Markov chains
	
	Markov processes are a special kind of stochastic processes with the characteristic of 
	having no memory. This means that only the current state of the process can influence
	where the process goes next \cite{norris1998markov}. In this work we focus on a special kind
	of Markov processes called Markov chains that can assume only a finite or countable
	number of states.
	
	Markov chains can be used to model many phenomena of interest and have been used in many
	applications, including [TODO]. What makes them particularly useful is that the memoryless
	property makes it feasible to predict future developments and compute probabilities and
	expected values to quantify that behavior.
	
	More formally, a Markov chain is a stochastic process $(X_t)_{t \ge 0}$ which can assume
	only a finite or countable number of states $i \in I$. The starting state
	is sampled from a probability distribution $\lambda$, called the initial distribution.
	We define $p_{ij}(t)$ to be the probability of the process being in state $j$ at time $t$
	when starting from state $i$ at time $0$ and a stochastic matrix $P(t)$, where $\left(P(i)\right)_{ij} = p_{ij}(t)$.
	Each row of $P(t)$ is thus a probability distribution over the state space $I$. In our
	work we assume that $P(t)$ is recurrent for every $t$ and that the process is non-explosive.
	
	The literature distinguishes between two types of Markov chains: discrete time and continuous time.
	Discrete time Markov chains are the most basic type of Markov chains, where the process
	changes states in discrete time steps $n \in \mathbb{N}$ with probabilities $p_{ij} = p_{ij}(1)$
	and obeys the Markov property presented in Definition \ref{thm:markov-property-discrete}.
	
	\begin{defn}
	\label{thm:markov-property-discrete}
	Let $(X_n)_{n \ge 0}$ be a discrete time Markov chain with initial distribution $\lambda$.
	Then conditional on $X_m = i$, $(X_{m + n})_{n \ge 0}$ is also a Markov chain with initial
	distribution $\delta_i$ and is independent of $X_0, X_1, ..., X_m$.
	\end{defn}
	
	Continuous time Markov chains are a generalization of discrete time Markov chains as they
	gap between the discrete time steps is filled and the process can change states at any given moment.
	Continuous time Markov chains are closely related to Poisson processes. Imagine the state
	space $I$ as a labyrinth of corridors and chambers. The corridor between chambers $i$ and $j$
	is shut by a single door that opens for an infinitely small amount of time at the jump times of
	a Poisson process with rate $q_{ij}$. If the person walking through the labyrinth changes
	chambers each time a door opens, they are performing
	a continuous time Markov chain.
	
	The basic data needed to define a continuous time Markov chain on state space $I$ are
	given by a transition rate matrix $Q$ satisfying the following conditions:
	
	\begin{enumerate}
		\item $q_{ij} \ge 0$ for all $i \ne j$
		\item $\sum_{j \in I} q_{ij} = 0$
		\item $-\infty < q_{ii} \le 0$ for all $i \in I$.
	\end{enumerate}
	
	Each off-diagonal element of $Q$, $q_{ij}$ represents the rate of jumping from state $i$ to state $j$,
	while the diagonal elements $q_{ii}$ represent the rate of leaving $i$. The stochastic matrix
	$P(t)$ is represented as $P(t) = e^{tQ}$ and can be computed by solving Kolmogorov's equations:
	\begin{equation}
		\frac{d}{dt}P(t) = P(t)Q
	\end{equation}
	with initial condition $P(0) = I$.

\end{itemize}